\section{Ruy Lopez, Closed, Zaitsev Variation}
\subsection{序論}
Ruy Lopez、とりわけClosed Variationは、常に人気のあるオープニングであり続けています。

ポジショナルな考え方とタクティカルな考え方が白黒双方に求められ、非常に複雑な局面が長く続くゲームになるため、あらゆるチェスの要素が詰め込まれていると言っても過言ではないでしょう。

その中でも、1990年代に非常に流行した形であり、世界選手権でも何度となく現れた形であるZaitsev Variation(9...Bb7)は、今でも新たな手が模索されているなど、常に進化を続けているオープニングでもあります。

今回から何回かにわたり、Zaitsev Variationの背後にある考え方を紹介していきたいと思います。

なお、底本はKuzminのThe Zaitsev System: Fresh Ideas and New Weapons for Black in the Ruy Lopezとしますが、私自身による研究(主にサイドライン)も含みます。


\subsubsection{Zaitsev Variationまでの手順}

やはりRuy Lopezについて書く際には、9. h3までの手順を簡単にでも解説しなければいけません。

{\bf 1.e4}

全ての始まりです。 

{\bf 1...e5 2.Nf3 Nc6 3.Bb5}

3. Bc4ならItalian、3. d4ならScotchになります。どちらも全く異なる展開になります。

{\bf 3...a6}

3... Nf6ならBerlin Defenseです。Kramnikが研究し、2000年の世界選手権に向けてKasparov対策として鍛え上げていった過去があります。 

{\bf 4.Ba4}

4. Bxc6(Exchange variation) 

{\bf 4...Nf6 5.O-O Be7}

5... Nxe4はOpen Variationと呼ばれます。Anandが1995年の世界選手権でKasparov対策としたのはこちら。

{\bf 6.Re1 b5}

大事な手で、これを入れずに6... O-O?は7. Bxc6から8. Nxe5とされます。

{\bf 7.Bb3 d6}

7...O-O 8. c3 d5も重要定跡です。Marshall Attackと呼ばれ、古くは1918年にMarshallがCapablancaに対して使った手です。最近では2004年の世界選手権で、LekoがKramnik対策として使いました。

{\bf 8.c3 O-O 9.h3}

この手を入れずに9. d4は9... Bg4です。白悪くはないと思いますが、あまり好まれません。 


\def\fena{r1bq1rk1/2p1bppp/p1np1n2/1p2p3/4P3/1BP2N1P/PP1P1PP1/RNBQR1K1 b - - 0 9}
\begin{center}
\chessboard[setfen=\fena]

9. h3まで
\end{center}

この9. h3までの形がRuy Lopez Closed Variationの基本形です。

\subsubsection{Zaitsev Variationまでの手順}

さて、9. h3以降、黒にはいくつかの手段があります。基本的に次の10. d4を防げないので、センターへ反撃する準備を整えますが、センターへの反撃形をどう作っていくかにより分岐します。

\begin{itemize}
    \item 9... Na5 (Chigorin Variation): ...Na5から...c5として、d4ポーンにプレッシャーをかけます。
    \item 9... h6 (Smyslov Variation): ...Re8から...Bf8として、e4ポーンにプレッシャーをかけますが、その際にf7が薄くなるためにNg5を事前にケアする意味です。
    \item 9... Nb8 (Breyer Variation): ...Nbd7、...Nc5としてセンターにナイトを置いたうえe4に圧力をかけて戦うプランです。
    \item 9... Bb7 (Zaitsev Variation): ...Bb7、...Re8の形を作り、e4にプレッシャーをかけます。
\end{itemize}

このように見ると、Smyslov、Breyer、Zaitsevはコンセプトとしては同じ、e4ポーンへのプレッシャーにより成り立っていることが見て取れます。

次に、それぞれのバリエーションで、もともとb8にいたナイトの動きを見てみましょう。

\begin{itemize}
    \item 9... Na5 (Chigorin Variation)は、ナイトをa5に跳ねています。このナイトは、この後c4あるいはc6に向かいます。
    \item 9... h6 (Smyslov Variation)は、ナイトの動きを保留しています。
    \item 9... Nb8 (Breyer Variation)は、ナイトをb8に戻し、この後d7-c5と使っていきます。
    \item 9... Bb7 (Zaitsev Variation)は、ナイトの動きを保留しています。
\end{itemize}

ナイトは、1手で行けるマスに2手かけて行くことができません(例えばルークなら、e1にいるルークは1手でe7に行けますが、e2-e7と2手かけて行くこともできます。そのため、ある局面でRe2と指した後にやっぱりe7が良かった、と思っても指しなおすことができます)。そのため、ナイトの動きは「局面を決定する(commiting)」動きであることが多いため、柔軟なプレーを行いたい場合にはナイトの動きは保留することがよい場合があります。

そうなると、Smyslov VariationとZaitsev Variationが、局面を柔軟に保つという意味では優れていると考えられます。では、Smyslov VariationとZaitsev Variationの違いは何でしょうか。

それを考えるために、Smyslov Variationの定跡を見ていきましょう。9. h3までは既知とします。

{\bf 9... h6 10. d4 Re8 11. Nbd2 Bf8 12. Nf1! Bd7 13. Ng3! }

\def\fenb{r2qrbk1/2pb1pp1/p1np1n1p/1p2p3/3PP3/1BP2NNP/PP3PP1/R1BQR1K1 b - - 0 13}
\begin{center}
\chessboard[setfen=\fenb]

13. Ng3まで
\end{center}

定跡手順に「!」を付けるのも変な話ですが、このようにナイトをf1-g3と動かすことで、e4ポーンを守るのがRuy Lopezの白番でよくある展開です。こうなってしまうと、e4ポーンは守られてしまうので黒はプランの変更を考えなければいけません。(そのためよくあるのは...Na5-...c5-...Nc6-...Ne7-...Ng6とするプランです)

ここでZaitsevの考え方が見えてきます。

\subsubsection{e4ポーンを取るために}
ここで黒が目を付けるのは、白のNf1の瞬間に、eポーンの守り駒がe1のルークだけになっているということです。

もともとe4ポーンを取りたいのがこのSmyslov Variationのコンセプトでした。では、取りに行ったらどうなるでしょうか?

{\bf 9... h6 10. d4 Re8 11. Nbd2 Bf8 12. Nf1! exd4?!}

\def\fenc{r1bqrbk1/2p2pp1/p1np1n1p/1p6/3pP3/1BP2N1P/PP3PP1/R1BQRNK1 w - - 0 13}
\begin{center}
\chessboard[setfen=\fenc]

12... exd4?!まで
\end{center}

一見、e4のポーンは落ちているように見えますが、

{\bf 13. cxd4! Nxe4?? 14. Bd5! +-}

\def\fend{r1bqrbk1/2p2pp1/p1np3p/1p1B4/3Pn3/5N1P/PP3PP1/R1BQRNK1 b - - 0 14}
\begin{center}
\chessboard[setfen=\fend]

14. Bd5!まで
\end{center}

これで白良しです。

ここで考えるのは、「e4ポーンを取りに行く手としては9... h6は役に立っていない」ということです。その1手を、代わりにc6のナイトを守ることに使えばNf1の瞬間にexd4が成立するのではないか?と考えるのは自然です。

やってみましょう。9... Bd7と9... Bb7がありますが、まずは9... Bd7から。

{\bf 9... Bd7 10. d4 Bf8 11. Nbd2 Bf8 12. Nf1 exd4!? 13. cxd4 Nxe4}

\def\fend{r2qrbk1/2pb1ppp/p1np4/1p6/3Pn3/1B3N1P/PP3PP1/R1BQRNK1 w - - 0 14}
\begin{center}
\chessboard[setfen=\fend]

13... Nxe4まで
\end{center}

タクティクス問題の局面になってしまいました。

{\bf 14. Rxe4!! Rxe4 15. Ng5! Re7? 16. Qh5 +-}

このように、b3にビショップがいるとf7を狙われて、e4を取る手は成立しません。

そのため、どこかで...Na5を入れたいのですが、今度はBc2と引かれてe4に数を足されて、e4を取ることができなくなります。

つまり、e4を取るためにはもう1枚e4に数を足さないといけないので……

\subsubsection{9... Bb7}

{\bf 9... Bb7!}

あえて「!」を付けます。これが良い手で、e4ポーンにもう1枚数を足します。

{\bf 10. d4 Re8 11. Nbd2 Bf8 12. Nf1?}

\def\fend{r2qrbk1/1bp2ppp/p1np1n2/1p2p3/3PP3/1BP2N1P/PP3PP1/R1BQRNK1 b - - 0 12}
\begin{center}
\chessboard[setfen=\fend]

12. Nf1?まで
\end{center}

{\bf 12... exd4! 13. cxd4 Na5! 14. Bc2 Nxe4! -=}

首尾よくeポーンを取れ、黒十分です。

さて、このようにNf1と引けないのであれば白は別の手を考える必要があります。

\subsubsection{Refutation 11. Ng5と"Refutation of Refutation"}

白はSmyslov Defense 9... h6という手を省くことで、e4ポーンを取ることを可能にしました。

しかし9... h6は、「f7が薄くなるためにNg5をケアする」手であったはずです。それでは、Ng5は大丈夫なのでしょうか?もしNg5が白の攻めとしてうまくいくのであれば、9... Bb7はその根本から不成立です。

結論から言いますと、Ng5は「白の攻めとして」は不成立です。(含みを持たせている理由は後程)

やっていきましょう。

{\bf 9... Bb7 10. d4 Re8 11. Ng5!?}

\def\fend{r2qr1k1/1bp1bppp/p1np1n2/1p2p1N1/3PP3/1BP4P/PP3PP1/RNBQR1K1 b - - 0 11}
\begin{center}
\chessboard[setfen=\fend]

11. Ng5!?まで
\end{center}

9... Bb7の考えを根本から問う手です。黒に受けは1つしかありません。

{\bf 11... Rf8}

白は攻め続けないと、黒に...h6とされて今度は白のほうが手損します。

{\bf 12. f4}

12. a4もありますが12... h6後メインラインに戻ります。こちらの方が独立したラインです。これが受かることが分かって、9... Bb7は市民権を得たと言えるでしょう。12... h6などでは13. Nf3とされて、白のセンターの圧力が強く、黒良くありません。

{\bf 12... exf4! 13. Bxf4 Na5 14. Bc2 Nd5!}

\def\fend{r2q1rk1/1bp1bppp/p2p4/np1n2N1/3PPB2/2P4P/PPB3P1/RN1QR1K1 w - - 0 15}
\begin{center}
\chessboard[setfen=\fend]

14... Nd5!まで
\end{center}

これでg5のナイトをアタックして、黒十分です。そのためこの手は、Zaitsev VariationのRefutationとなりうる11. Ng5に対してのRefutationという意味で、Kuzmin本では"Refutation of Refutation"と呼ばれています。

さて、これが成立しないのであれば11... Rf8には12. Nf3と戻るしかなく、黒はZaitsev Variationを指し続けるのであれば12... Re8と指すため、ドローになります。

これが「白の攻めとして」は不成立であるといった理由であり、白からドローにすることができます。

黒は12... h6などもありますが、これはSmyslov Variationのやや劣る変化(..Bd7ではなく...Bb7と指した変化)に合流するため、あまり好まれていません。

そのため別の変化(PogoninaやKovanovaなど、ロシア女子チームにより研究された変化であり、Kuzmin本ではSaratov Variationと呼ばれています)に合流するなど、研究が進められています。

ここまでがZaitsev Variationの序章です。次章以降は各変化を掘り下げていきます。


\subsection{12. a3 (Sochi Variation)}

{\bf 1. e4 e5 2. Nf3 Nc6 3. Bb5 a6 4. Ba4 Nf6 5. O-O Be7 6. Re1 b5 7. Bb3 d6 8. c3 O-O 9. h3 Bb7 10. d4 Re8 11. Nbd2 Bf8}

\def\fend{r2qrbk1/1bp2ppp/p1np1n2/1p2p3/3PP3/1BP2N1P/PP1N1PP1/R1BQR1K1 w - - 3 12}
\begin{center}
\chessboard[setfen=\fend]

基本図
\end{center}

\subsubsection{12. a3の考え方}
12. a3は、黒の陣形にプレッシャーをかける手ではありません。それでは、12. a3の狙いは何でしょうか。

まず一つは、b3にいるビショップがアタックされた際に、a2に引き場所を残すという意味があります。a2-g8ダイアゴナルにビショップを残すことで、f7の点、および黒キングへの睨みを残しておくという意味があります。

もう一つは、b4の地点をコントロールしておくという意味です。今後紹介する12. a4変化では、b4の地点がやや弱体化するうえ、黒のナイトがb4に入ってくる変化もあります。そのような変化を避けるためにも、b4地点のコントロールは残しておきたいという考え方があります。

\subsubsection{最近の実戦例}
それでは、最新の実戦例を見ていきます。Topalov-Mamedyarovの、2019年Gashimov記念大会からのゲームです。

{\bf 1. e4 e5 2. Nf3 Nc6 3. Bb5 a6 4. Ba4 Nf6 5. O-O Be7}

MamedyarovはOpen Spanishを多く指すので意外と言えば意外な選択です。

{\bf 6. Re1 b5 7. Bb3 d6 8. c3 O-O 9. h3 Bb7 10. d4 Re8 11. Ng5 Rf8 12. Nf3 Re8 13. Nbd2}

Ng5-Nf3の往復が入っているので、総手数が2手多くなります。

{\bf 13... Bf8 14. a3}

今回テーマとする局面です。

{\bf 14... g6}

多いのはここで14... h6(通常は12... h6ですが)ですし、KuzminのZaitsev本でも12... h6を取り扱っています。ビショップがa2に引けるため、将来的にNg5の余地を残すのは危険という考え方でしょうか。

{\bf 15. Ba2 Bg7 16. b4}

Zaitsev Variationは、黒のクイーンズナイトの動きが重要です。Nc6-Nb4が一つのオプション。Nc6-Nb8-Nd7-Nc5がもう一つのオプションです。a3-b4のポーン形は、その両方のオプションを不可能にしています。

もちろん良いことだけではなく、黒のa6-b5のポーン形に対して白がa4突きから手を作れるのと同様、黒もa5突きからこのポーン形に対して手を作っていけます。

さらにその時に、ナイトがc6に残っていることで...a5がb4取りの狙いになっていることが黒の有利に働きます。

{\bf 16... exd4 17. cxd4 a5}

\def\fend{r2qr1k1/1bp2pbp/2np1np1/pp6/1P1PP3/P4N1P/B2N1PP1/R1BQR1K1 w - - 0 18}
\begin{center}
\chessboard[setfen=\fend]

17... a5まで
\end{center}

上記のプランです。

18. Rb1

GM Bojkovの解説によれば、この手が新手であり、従来の18. Qb3では黒問題なし、とのことです。(出典: https://www.chess.com/news/view/shamkir-chess-gashimov-memorial-round-6)

ただ、本譜のラインも黒はイコアライズに成功するので、白は18手目で他の手段を探さないといけません。

ちなみにコンピュータはここで、衝撃の18. d5 Nxd5!? 19. exd5 Rxe1+ 20. Qxd1 Nxb4! 21. axb4 Bxa1というラインを推奨します。これで白やや指しやすいという判断のようです。

{\bf 18... axb4 19. d5 Ne5 20. Nxe5 dxe5 21. Rxb4 c6 22. dxc6 Bxc6 23. Qf3 Bf8}

Topalovによれば黒問題なしと思っていた、とのことです。

{\bf 24. Rb1 Ra4 25. Nf1 Rxe4}

黒はポーンを取れましたが、ここからのTopalovの反撃は彼の全盛期を思い出させます。1手ごとにスレットを作り、駒得やメイトに持ち込むという、主導権を最大限に活用したプレーです。

{\bf 26. Rd1! Qe7 27. Bg5}

g5のマスを白の攻撃の拠点にします。ナイトをピンしてターゲットにします。

{\bf 27... Bg7 28. Ne3 Qxa3 29. Ra1!}

直接的にはBxf7+の狙い。

{\bf 29...Qc5 30. Rdc1 Rc4}

黒も反撃として、クイーンをディスカバードアタックします。

{\bf 31. Qd1}

自分の狙いが残っているときには、相手の狙いを落ち着いて受けても十分なことがあります。お互いに狙いを実現しあって結果的に自分が優勢になることが見えていればいいのですが、えてしてそのような手順は非常に複雑で、読む量も多くなります。

{\bf 31... Ne4 32. Bxc4 bxc4 33. Rxc4 1-0}

このゲームを分析して思うことは、
\begin{itemize}
    \item a3-Ba2ラインでは、黒は、白のg5マスへのビショップ・ナイトの侵入に気を付けていないといけない。
    \item 黒はNb8-Nd7-Nc5のナイトの再配置を狙うが、白にb4突きで妨害されたときには白のポーン形を...a5から崩すことも視野に入る。
\end{itemize}
ということです。

\subsection{12. d5 (Modern Variation)}
12. d5はセンターの緊張を解消するので、Zaitsevの変化の中ではかなり穏やかに進む変化です。しかし白には明快なポジショナルなプランがあり、黒は決して侮ってはいけません。

黒の基本プランとしては、12... Nb8からNd7-Nc5というルートでナイトを転戦させます。加えて、...c6からcポーンと白のdポーンを交換することで、センターのポーンのマジョリティを主張します。

一方白の基本プランは、d5のホールを活用することです。そのために、白は黒の駒のうち、白マスに効く駒(ナイト2つ、白マスビショップ)を消すように指します。

実際のゲームを見ていきましょう。2018年ジブラルタルオープンでの、Navara-Oparinでのゲームです。

{\bf 1. e4 e5 2. Nf3 Nc6 3. Bb5 a6 4. Ba4 Nf6 5. O-O Be7 6. Re1 b5 7. Bb3 d6 8. c3 O-O 9. h3 Bb7 10. d4 Re8 11. Nbd2 Bf8}

ここまでは前回と同じなので割愛します。

{\bf 12. d5}

\def\fend{r2qrbk1/1bp2ppp/p1np1n2/1p1Pp3/4P3/1BP2N1P/PP1N1PP1/R1BQR1K1 b - - 0 12}
\begin{center}
\chessboard[setfen=\fend]

12. d5まで
\end{center}

今回のメインムーブです。センターを閉じ、スペースを主張します。

{\bf 12... Nb8 13. Nf1 Nbd7 14. N3h2}

この手は覚えておくべき手です。一見、14. N1h2のほうが自然ですが、本譜と同じように進んだ時に途中で違いが出てきます。

{\bf 14... Nc5 15. Bc2 c6}

\def\fend{r2qrbk1/1b3ppp/p1pp1n2/1pnPp3/4P3/2P4P/PPB2PPN/R1BQRNK1 w - - 0 16}
\begin{center}
\chessboard[setfen=\fend]

15... c6まで
\end{center}

黒は、白のセンターポーンを崩し、センターポーンの多さを主張できる形にします。

{\bf 16. b4}

大事な挿入手で、先にdxc6 Bxc6の交換を入れると、b4に対して...Nxe4が可能になります。

{\bf 16... Ncd7 17. dxc6 Bxc6 18. Bg5}


白のポジショナルなプランの第一段階です。f6のナイトを消すことで、黒のd5支配を弱めます。

{\bf 18... h6 19. Bxf6 Nxf6}

新しいナイトがf6に補充されますが、

{\bf 20. Ng4}

そのナイトも消してしまいます。ここで、14. N1h2と指していた場合には、20... Nxg4 21. hxg4となり少し形が崩れます。

{\bf 20... Nh7}

従来はここで20... Nxg4が多い形でした。しかし、Oparinが指したこの手がかなり有力であるようです。この手の狙いは、いったんナイトを退避させておいて、白のナイトにg4を与えないような駒の組み換えをした後にナイトをf6に戻すことです。

20... Nxg4だと、白マスビショップの交換後、白はd5に2手で飛べるナイトを持っているのに対して黒がd5に効かない黒マスビショップを持つエンドゲームになり、そのような展開を嫌うプレーヤーがかなり多いと思います。

他には20... a5も有力です(Caruanaが指しています)。

{\bf 21. Qd3 Rc8 22. Bb3 Re7 23. c4 h5}

これが、「白のナイトにg4を与えない駒の組み換え」です。もちろんhポーンが弱くなるという代償も黒にはあります。

{\bf 24. Nge3 Nf6}

これで白はe4を守らないといけないので、

{\bf 25. Nd2}

一旦ナイトをd2に運びます。しかしこれで、白のナイトからd5が遠くなったので、

{\bf 25... Rb7 26. a3 g6 27. Rad1 Bh6 28. cxb5 axb5 29. Nd5 Bxd5 30. Bxd5 Nxd5 31. Qxd5}

\def\fend{2rq2k1/1r3p2/3p2pb/1p1Qp2p/1P2P3/P6P/3N1PP1/3RR1K1 b - - 0 31}
\begin{center}
\chessboard[setfen=\fend]

31. Qxd5まで
\end{center}

このように総交換しても、白はすぐにd5を使えないという主張ができます。

{\bf 31... Qb6 32. a4!?}

そのため白はナイトをc4-d6の経路で使いますが、その間に黒はカウンターを作ることができます。

{\bf 32... bxa4 33. Nc4 Qa7 34. Nxd6 Rc2! 35. Rf1 Rxb4}

これでアウトサイドにパスポーンを作ることができ、黒が少し指しやすくなります。

{\bf 36. Qxe5 Qc7 37. Qe8+ Bf8 38. e5 Qe7 39. Qxe7 Bxe7 40. Rc1 Rxc1 41. Rxc1 Rb8 42. Rc7 Ra8 43. Rxe7 a3 44. e6 fxe6 45. Rxe6 a2 46. Re1 a1=Q 47. Rxa1 Rxa1+ 48. Kh2 Ra2 49. Kg3 Kg7 50. Ne4 Kf7 51. h4 Ke6 52. Kf4 Ra4 53. g3 Rb4 54. Kf3 1/2-1/2}

最後はドローになりました。

この手順は、白黒ともにプランがわかりやすく、どちらも指しやすいのではないかと思います。12. a4がメインラインであることは確かですが、それでも十分研究する意味のある定跡であると思います。

\subsection{12. Bc2 (Karpov/Geller Variation)}

\subsubsection{白のプラン}
Kuzmin本によれば、a4を突かない12. Bc2の狙いは、b2-c4-d5-e4という閉じたセンター形を作ることです。この形についてはKasparovもRevolution in the 70'sで書いており、Kasparovによれば12. Bc2は「e4ポーンをサポートしスペースを確保するという、最も理解しやすいプランである」とされています。

場合によってはクイーンサイドを開くa4突きとも組み合わせて指します。

\subsubsection{黒のプラン}
黒がまず考えたいことは、白のビショップがb3-g8のダイアゴナルから外れたため、白Ng5の脅威が薄れているということです。

そのため、Ng5を防ぐための...h6を指さず、12... g6と指してf8のビショップを活用していくことも考えられます。

しかしながら、12... g6に対して白は13. d5 Nb8 14. b3 c6 15. c4のようにして黒のg6突きを無駄手にする変化があります(Kasparov、Kuzmin)。その局面は評価が難しいところです。

他にも、ポーンを動かさず12... Nb8とする手もあります。しかし、KarpovがBeliavskyに指したように13. a4! から 14. Bd3!とb5ポーンにプレッシャーをかける手が強く、これは白良しと考えられています。

3つめの可能性が12... h6です。Morozevichが良く指しています。

\subsubsection{実戦例}
Leko - Morozevich (Biel, 2017)を紹介します。

{\bf 1. e4 e5 2. Nf3 Nc6 3. Bb5 a6 4. Ba4 Nf6 5. O-O Be7 6. Re1 b5 7. Bb3 d6 8. c3
O-O 9. h3 Re8 10. d4 Bb7 11. Nbd2 Bf8 12. Bc2 h6}

白に次のようにセンターを固めさせる手を誘います。

{\bf 13. d5 Ne7 14. b3 c6 15. c4}

\def\fend{r2qrbk1/1b2npp1/p1pp1n1p/1p1Pp3/2P1P3/1P3N1P/P1BN1PP1/R1BQR1K1 b - - 0 15}
\begin{center}
\chessboard[setfen=\fend]

15.c4 まで
\end{center}

こう見ると、センターを固めたことで黒の12 ...h6がキングサイドを弱めただけの手になっているように見えますが、黒には15... cxd4 16. cxd4 Nd7から...f5を狙うカウンターの手段が生じています。

King's Indianによくあるように、...f5に対しては白Ng5-Ne6が狙いになりますが、この場合は先に...h6を入れることでNg5の反撃を防いでいます。

しかしながら、16... Nd7後、17. a4! f5 18. b4!(Kuzmin)が良い手で、白が主導権を握れるとされています。おそらくそのため、Morozevichはその変化を避けました。

{\bf 15... a5}

Morozevichの改良手段でしょう。先のKuzminの変化では、18. b4!の後Bb2からダイアゴナルを抑える手が好手と考えられています。そのため、a,bファイルを開け、特にbファイルを支配することでその手を避ける狙いがあると考えられます。

{\bf 16. Nf1 a4 17. Rb1 axb3 18. axb3 Qc7 19. Ne3}

センターをサポートすると同時に、黒の...f5を防いでいます。

{\bf 19... bxc4 20. bxc4 Reb8}

黒はa,bファイルからプレッシャーをかけていきます。

{\bf 21. Bd2 Bc8 22. Rxb8 Qxb8 23. Qb1 Qc7 24. Qb2 Ng6 25. Ra1 Rxa1+ 26. Qxa1 Nf4 27. Kh2 cxd5 28.cxd5 Qb6}

\def\fend{2b2bk1/5pp1/1q1p1n1p/3Pp3/4Pn2/4NN1P/2BB1PPK/Q7 w - - 1 29}
\begin{center}
\chessboard[setfen=\fend]

28... Qb6 まで
\end{center}


ポーン形はほぼ対称で、黒の陣形もスペースは狭いですがコンパクトです。総じて序盤は成功しているといえるでしょう。この後はミドルゲーム、エンドゲームの領域に入るため本連載の領域を逸脱するので簡便に書きます。

{\bf 29. Qa5 Qa6 30. Qxa6 Bxa6 31. Ne1 h5 32. f3 g6 33. Bb3 Nd3 34. Bc4 Bxc4 35. Nxc4 Nxe1 36. Bxe1 Ne8 37. Bb4 f6 38. g4 hxg4 39. fxg4 f5 40. exf5 gxf5 41. gxf5 Nf6 42. Bxd6 Bxd6 43. Nxd6 Kf8}

わざと2ポーンダウンにして白のポーンの連携を切ってドローを取りに行きます。ドローは取れる形のようですが、かなりテクニックが必要のように見えます。

{\bf 44. Kg3 Ke7 45. Nc8+ Kd7 46. Nb6+ Kd6 47. Kh4 Kc5 48. Na4+ Kd4 49. d6 e4 50. Kg5 Nd7 51. Kf4 e3 52. Kf3 Ke5?}

52... Ne5+!であればドローのようですが、非常に難しいです。ポイントは黒のナイトをe5に置いておくことで、白Kxe3の時にNc4+!のフォークでe6ポーンを落とす手を見せ、その手を防ぐために白Nb6を強要し、白ナイトが1手でe4に来られない位置に跳ねたのを見てKxf5-Ke6-Kxd6としてポーンを回収することのようです。本譜を見ると白ナイトがe4に行くことの重要性がわかります。

{\bf 53. Kxe3 Kxf5 54. Nc3 Ke6 55. Ne4!}

ナイトはパスポーンの後ろから、という原則に従っています。これでd6を守って白勝勢です。

{\bf 55... Ne5 56. Kf4 Nc4 57. Kg5 Ne5 58. h4 Nf3+ 59. Kg4 Ne5+ 60. Kf4 Ng6+ 61. Kg5 Nf8 62. Kh6 Ke5 63. Kg7 Ne6+ 64. Kg8 Kxe4 65. h5 Ke5 66. h6 Kd5 1-0}


\subsection{12. a4 (Kasparov Variation)}
この変化は極めて難解かつ相当量の記憶を求められる変化で、今ではSuperGMの間で白が12. a4を避ける事態となっています(Kuzmin)。

というわけで12. a4に関してはKuzmin本に書かれている定跡の歴史と、一つ興味深いゲームを紹介するにとどめます。

\subsubsection{歴史}
Zaitsev Variationに対する12. a4は、データベースを紐解くと重要対局で最初に指したのはKasparovと出てきます。(Kasparov-Dorfman、1978年ソ連選手権決勝トーナメント、白勝ち)。

Kuzmin本によれば、ZaitsevがZaitsev Variationのアイディア(...h6を入れずにe4にプレッシャーをかけ、白Ng5からf4に対して...Nd5!で反撃する)を思いついたのが1970年、Zaitsev Variationが重要局で見られ始めるようになったのが1975年ですから、この定跡の最初期から12. a4はあったということになります。そして、この12. a4をZaitsev Variationに対してその後多用するのも、ほかならぬKasparovです。

さて、Zaitsevと言えば言わずと知れたKarpovのセコンドです。必然的に、この変化はKasparov-Karpovの世界選手権、通称KK Matchで繰り返し使われることになります。

最初期はDorfmanに対してKasparovが指したように、d5-c4とポーンを突いていくプランで白は指していました。しかしのちにそれが上手くいかないことがわかり、センターの緊張を保ったままの12. a4 h6 13. Bc2に切り替えます。

一方黒も、13. Bc2に対する対処は時代により変遷してきました。より古いBreyer Variation風の13... Nb8は、1985年の2回目のKK Matchで使われた手ですが、良くないとされています。

1986年の3回目のKK Matchでは、黒番Karpov側からの改良、13... exd4 14. cxd4 Nb4 15. Bb1 c5 16. d5 Nd7 17. Ra3 c4が指されました。Karpovはこの変化で負けましたが、これは今日でも主要な変化の一つとされています。

さらに時は下って1990年のニューヨーク・リヨンでの5回目のKK MatchではKarpov側からの改良手段が出されました。

余談ですがKuzmin本には、1990年の世界選手権のオープニングセレモニーにミロス・フォアマン(「アマデウス」と「カッコーの巣の上で」の映画監督)が招かれており、ちょうどKuzminがいた隣のテーブルでカルポフと話していたことが書かれています。(フォアマンの大ファンだった)Kuzminはオープニングセレモニーの数日後にKarpovの妻と話していてそのことを知ったようでしたが、当時のチェスの注目度の高さがうかがい知れます。

話題を戻すと、1990年の世界選手権では12... h6 13. Bc2 exd4 14. cxd4 Nb4 15. Bb1以降、15... bxa4 15. Rxa4 a5 17. Ra3と進む変化と、15... c5 16. d5 Nd7 17. Ra3 f5と進む変化が試されました。このうち前者は今ではほとんど指されませんが、後者は生き残りました。

これらの世界最高の研究家による検討の結果、最終的に最も信頼できる黒の変化は17... c4ということになっています。今でもGMレベルで研究が進められている形です。

\subsubsection{17... c4に対する白の手段}
白もいくつかの手段が試されてきました。

まず、KK Matchで試されたのが18. Nd4と、18. axb5 axb5 19. Nd4です。このうち前者に対しては黒も18... Qf6でだいぶ良く戦えているようです(詳述はしません)。一方後者の方がより黒は対処が難しく、今日では19... Rxa3 20. bxa3 Nd3 21. Bxd3 cxd3以降難解な戦いが続いています。

近年注目を浴びているのが18. Ree3です。最初に見たときには18. Rae3の書き間違いではないかと思ったほど変わった手ですが、黒はこの手に対して対処するのが難しいようです。実際にゲームを見ていきます。

\subsubsection{実際のゲーム(Pijpers-Habu, 2016)}
Kuzmin本にも一変化として載っているゲームを1つ取り上げます。黒番HabuはもちろんYoshiharu, Habuです。

{\bf 1. e4 e5 2. Nf3 Nc6 3. Bb5 a6 4. Ba4 Nf6 5. O-O Be7 6. Re1 b5 7. Bb3 d6 8. c3 O-O 9. h3 Bb7 10. d4 Re8 11. Nbd2 Bf8 12. a4 h6 13. Bc2 exd4 14. cxd4 Nb4 15. Bb1 c5 16. d5 Nd7 17. Ra3 c4 18. Ree3!}

トップレベルでは2010年に初めて指された手で、黒はこの手に対する対処を見つけないといけないと言われています。

白はNd4のプランがあり、そうした時にルークをg3に回すことができます。3段目をルークの展開に使うプランは良く見られますが、このように2つのルークを3段目で使うのはなかなか珍しい手です。

\def\fend{r2qrbk1/1b1n1pp1/p2p3p/1p1P4/Pnp1P3/R3RN1P/1P1N1PP1/1BBQ2K1 b - - 0 18}
\begin{center}
\chessboard[setfen=\fend]

18. Ree3 まで
\end{center}

{\bf 19... Nc5}

e4ポーンにプレッシャーをかける手です。しかし、(axb5の後)b5とb4が浮くため、黒はこちらの攻撃にも対処しなければいけません。もちろん、Closed Spanish全般における注意点として、白のキングサイドアタックも相当強いため、盤面全体で戦いが起きる形になります。

{\bf 19. b3!?}

攻撃側の方針として、ラインを開けるということがあります。この手はその方針に従った手です。

{\bf 19... cxb3}

この一見何でもないようなポーン取りで、白が一気に良くなるようです。より信頼できる手は19... c3! 20. Rxc3 f5!と、ポーンを捨ててルークをeファイルから動かし、その後f5を突くという手順で、いわゆる「1歩で1手を稼ぐ」手と言えるでしょうが、そのような手がチェスの、しかも序盤で出るのは非常に珍しいと言わざるを得ないでしょう。

{\bf 20. Nxb3 Nxa4 21. Nfd4}

Kuzmin本では「これにて白良し」と書かれています。

{\bf 21... Rc8 22. Bd2}

黒からの...Nc3!を防ぎながらb4のナイトに当ててテンポを取る手で、これでクイーンサイドのカウンタープレーを消し、キングサイドから攻めます。黒はナイトを守っていると、23. Rg3!から一気に白の攻めが決まります。

{\bf 22... Nxd5 23. exd5 Bxd5}

駒割はこれでナイト対3ポーンですが、白の主導権が非常に強い形です。以降、白がよどみなく攻め切りました。

{\bf 24. Rg3 Be4 25. Bxe4 Rxe4 26. Bxh6 Nc3 27. Qa1 Qe8 28. Bd2 b4 29. Rxa6 Qe5 30. Re3 d5 31. Ra8 Rc4 32. Qa7 1-0}

\subsection{Saratov Variation}

さて、今回はZaitsev Variationの番外編として、Saratov Variationと呼ばれる変化を取り上げます。

Zaitsev Variationを、黒番を持って指すうえで気を付けなければならない点として、11. Ng5から12. Nf3と指された際に、黒は純粋なZaitsev Variationを指そうとするとドローになる、という点が挙げられます。

この手順を避けるために12... h6と指すと、Smyslov Variationの一変化になりますが、この変化をあまり好まないプレーヤーもいます。(非常に細かいところですが、Smyslov Variationでは黒がb5のポーンを守るために白マスビショップをd7に展開する流れがよく見られます)

というわけで、その局面での第三の選択肢としてKuzmin本に書かれているのがSaratov Variationです。

そこまでの手順を見ていきましょう。

{\bf 1. e4 e5 2. Nf3 Nc6 3. Bb5 a6 4. Ba4 Nf6 5. O-O Be7 6. Re1 b5 7. Bb3 d6 8. c3 O-O 9. h3 Bb7 10. d4 Re8 11. Ng5 Rf8 12. Nf3 Nd7!?}

\def\fend{r2q1rk1/1bpnbppp/p1np4/1p2p3/3PP3/1BP2N1P/PP3PP1/RNBQR1K1 w - - 5 13}
\begin{center}
\chessboard[setfen=\fend]

12... Nd7 まで
\end{center}

この変化を、KuzminはSaratov Variationと呼んでいます。Saratovはロシア南部の都市で、この都市出身のGM Pogoninaが指したことで、この名前で呼んでいるようです。

10... Nd7でも同じ局面になりますが、Kuzminはあくまでこの変化に関して「Zaitsev Variationを白が避けたときに指す変化である」と書いており、定跡としての強さはZaitsev Variationのほうが強いという考えのようです。そのため、白が11. Ng5-12. Nf3としてZaitsev Variationを避けた時のみこの変化を使うことを推奨しています。

ここからの黒の指し手は非常にわかりやすいでしょう。白に強いセンターを許す代わりに、d4, e4のポーンをターゲットにして指していきます。その意味では、ハイパーモダン風な指し方に近いかもしれません。

{\bf 13. Nbd2 exd4 14. cxd4 Bf6! 15. Nf1 Na5 16. Bc2 Re8}

ここで白の手が広く、自然な17. Ng3、黒のNa5-Nc4に対する予防手17. Rb1、ピース配置を改善する17. N1h2、黒のc5突きを防ぐ17. Bf4などが指されています。

白がセンターを抑えているため、黒は白のキングサイドアタックに常に気を付けないといけませんが、それをしのいでしまえばクイーンサイドから十分反撃できる形になります。
