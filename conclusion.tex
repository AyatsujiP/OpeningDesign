\section{おわりに}

本書では、1. e4に対する基本的な4つの応手(1... c6, 1... e6, 1... c5, 1...e5)に対する特定の応手に限定して、序盤定跡の歴史と変遷を追う形で紹介を行いました。
注意しないといけないことは、これでもまだ定跡のほんの一部を紹介しているにすぎない、ということです。本書だけを読んでトーナメントに臨むことは、歴史小説を読んで歴史の試験に臨むようなものです。一部は深く掘り下げられていますが、広い知識は得られません。ぜひとも、幅広く知識を身に着けていただきたいと思います。

定跡を研究するということは、絶え間ない発見の連続です。データベースを検索し、その時々の流行りのラインを調べ、あるラインの評価を変えたであろうゲームを見つけた時の喜びは計り知れません。そのような意味では、定跡を研究するということは考古学に近いといえるかもしれません。
私は、定跡の研究において最も必要な能力は、子供が遊ぶ「間違い探し」のような2つの似通ったポジションから、異なった結論を引き出す力だと思います。Aのラインの駒配置とBのラインの駒配置の違いは何か、その違いは何をもたらすか、ということを考え、そして結論を引き出すことが定跡の研究において必要です。

もしこの本を読んで、自分も定跡の研究をしたいと考えた方がいれば、ぜひともはじめてみることをお勧めします。

\begin{flushright}
2019年10月\\
antilles
\end{flushright}