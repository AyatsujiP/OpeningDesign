\section{Sicilian, Najdorf, Scheveningen Formation}
\subsection{序論}
Sicilian DefenseのNajdorf Variationは極めて有名なオープニングで、1. e4に対して勝ちに行くための黒の序盤として、トップGMからクラブプレーヤーまで幅広く指されています。黒番で1. e4に対してNajdorfを指すプレーヤーは日本にも多いと思います。

定跡は極めて複雑ですが、黒としてはある意味では「目指すべき形」が明確であり、覚えやすい定跡ともいえると思います。

さて、Najdorfプレーヤーにとって必ず対策すべきなのが、6. Be3 (English Attack)でしょう。

GM NunnやGM Shortらの研究による、DragonのYugoslav Attackの攻め筋をNajdorfで使ったらどうなるか?という問いから始まった定跡ですが、Be3-f3-g4とする攻めの形がわかりやすく、Najdorf対策にEnglish Attackを採用しているプレーヤーも多いと思います。

English Attackに対する黒の対策は、人それぞれだと思いますが、6... e5が多いと思います。しかし私は6... e6に好感触を持っています。

本章では、黒の立場から、English Attackに対して6... e6を指す際に何を考えるか、どのように指していけばよいかを紹介したいと思います。


\subsubsection{1923 Scheveningen大会}

Scheveningen(スヘフェニンゲン)とはオランダ、デン・ハーグのリゾート地の地名です。この地で1923年、チェスの大会が開かれました。この大会で、e6とd6にポーンを並べる、Scheveningen Variationが指されたといわれています(Maroczy-Euwe, 1923)。

この大会の後、徐々にSicilian Defenseにおいてd6, e6にポーンを並べる形が流行するようになります。

当時のよくあるScheveningen Variationは1. e4 c5 2. Nf3 Nc6 3. d4 cxd4 4. Nxd4 Nf6 5. Nc3 d6 6. Be2 e6などのようにしてセンターにポーンを並べる形でした。これに対して白はe4, f4にポーンを並べ、白マスビショップをBe2-Bf3と使って戦っていました。

\subsubsection{Keres Attackの登場(1943)}

Scheveningen Variationは、一定の評判を保っていましたが、そこに衝撃的な新手が現れます。

1943年のSalzburg大会でKeresが指した手で、現在ではKeres Attackと呼ばれます。これは、1. e4 c5 2. Ne2(Nf3でもいずれ同じです) d6 3. d4 cxd4 4. Nxd4 Nf6 5. Nc3 e6に対して6. g4!と指す形です。

余談ですが、1943年は独ソ戦の真っ最中です。ソ連のプレーヤーとして知られるKeresが、なぜ(現オーストリア領)ザルツブルクのトーナメントに出られたのか?と思って調べてみたところ、まずそもそもKeresはエストニアのプレーヤーであり、エストニアは1941年からドイツ占領下にあったそうです。ちなみに当時フランスにいたAlekhineもこのトーナメントに参加しています。

5... e6によってc8のビショップの効きが止まるためにこのポーン突きが可能になります。そしてすぐにg5まで延び、f6のナイトを脅かしつつキングサイドにプレッシャーをかけていきます。このように、キングサイドを押し上げるプランが、Scheveningen型のSicilianに対してこの後有効だと認識されていきます。

\subsubsection{Najdorf Variationの登場(1950年代)}

さて、その後Sicilianの黒番でも、白番でも多くの新しいアイディアが生まれました。とりわけ最も重要なアイディアは、なんといっても5... a6(Najdorf Variation)でしょう。この一見手待ちにしか見えない(Fischerでさえ、「60」の中で5... a6を「手待ち」と言っています)手が、黒のb5突きを準備し、白からのNdb5を防ぎ、ある変化においては黒がRa7と指せるようになるなど、極めて多くのアイディアを見据えた手として多くのプレーヤーに愛されるようになります。

\subsubsection{English Attackの登場(1980年代)}

Najdorfに対する白の対策も進化してきました。最初は6. Be2が多かったですが次第に6. Bg5が増え、1980年代には6. Be3(English Attack)も見られるようになりました。のちにQd2からO-O-Oとしてキングサイドを攻める手を見せつつ、クイーンサイドにも目を光らせ、6... e6に対して7. g4!?の可能性も残した柔軟な手です。

\subsubsection{KasparovとScheveningen Formation(1980年代)}

さて、若きKasparovも黒番でScheveningenを愛用するプレーヤーでした。Kasparovは若いころは、Scheveningen型に組んでセンターを受けた後、a6, b5としてクイーンサイドを押していくプランを採用し、黒番での勝ちを重ねていきました。Scheveningen型が攻撃力を秘めているということを明らかにしたのはKasparovといっていいように思います。

一方、Kasparovのライバルとして知られたKarpovは、Keres Attackが大得意。Keres Attackは、白が主導権を握ることが多くなるため、a6, b5のプランを黒が取ることは難しいです。面白いことに、Karpov-KasparovでKeres Attackになった対局は1つしかないようです。お互いに相手の得意形を避けたということでしょうか?

Positional Playerとして知られるKarpovがKeres Attackが得意というのも面白い話ですが、KarpovのPositional Playは盤面全体に圧力をかけて相手の動きを奪うような指し方ということもできます。その意味では、Keres Attackからキングサイドのポーンを伸ばしていくプランは、ある意味ではKarpovのプレースタイルに合っているともいえるでしょう。

そこで、Kasparovは、いずれ後に必ず指すであろうa6を先に指し、そこからe6, b5と続ける指し方を採用しています。5... a6に対して6. g4??は指せないので。

この指し方がKasparovの専売特許かどうかは調べられませんでしたが、ここに至って、クイーンサイドアタックのための「Najdorf(5... a6)のScheveningen Formation(6... e6)」はいったんの完成を見たと言えるでしょう。

一方で白は、黒の手には関係なくEnglish Attack Formation(Be3 - Qd2 - O-O-O)を取ることができます。これで、5... a6 6. Be3 e6という手順の歴史が紐解けました。

\subsection{局面のポイント}
まずは、局面のポイントと白黒双方のテーマを考えていきたいと思います。

それではこの局面から。6. Be3のEnglish Attack型を想定します。

{\bf 1. e4 c5 2. Nf3 d6 3. d4 cxd4 4. Nxd4 Nf6 5. Nc3 a6 6. Be3 e6 7. f3 b5}

\def\fena{rnbqkb1r/5ppp/p2ppn2/1p6/3NP3/2N1BP2/PPP3PP/R2QKB1R w KQkq - 0 1}
\begin{center}
\chessboard[setfen=\fena]

7... b5まで
\end{center}

このようなポジションでよく出てくる手を考えていきます。英語では"Thematic"と呼ばれますが対応する日本語が思い当たらないです。将棋の解説だとよく「習いある手筋」と言ったりしますが、それに近いでしょうか。

\subsubsection{局面評価}

手を考える際に、まずは局面の評価をしていきます。
\begin{enumerate}
\item 白のメリット: キングサイドに広いスペースが確保できる。駒の展開が早い。そのためキングサイドで攻勢を取れる。
\item 黒のメリット: クイーンサイドにスペースが確保できる。センターポーンが白より1つ多い。そのため白のサイドアタックに対してセンターから反撃できる。また、終盤になったときにポーン形が白よりも良い。
\end{enumerate}

どちらにも主張がある局面です。どちらかというと白にダイナミックな主張が多く、黒の主張はポジションの良さなどスタティックなものともいえるでしょう。

次にポーン形を見てみます。
\def\fena{4k3/5ppp/p2pp3/1p6/4P3/5P2/PPP3PP/4K3 w - - 0 1}
\begin{center}
\chessboard[setfen=\fena]

\end{center}

このポーン形から、ポーンの弱点を考えていきます。
まず弱点に見えるのが、黒のd6ポーンです。次に白のc2ポーンも、セミオープンファイルにあるポーンとして弱点に見えます。
が、この形ではしばしば、白は黒のbポーンの伸びすぎ、あるいはe6ポーンへのサクリファイスを狙い、黒は白のe4ポーンが浮くのを狙う、というテーマが見られます。
d6とc2のポーンは、かなり固く守られているので、少なくとも中盤では攻撃目標になることは少ないです。

\subsubsection{白のテーマ}

白のプランとしてまず考えられるのがキングサイドアタックです。ポーンを突いていき(ポーン・ストームといいます)、キャスリングした黒のキングを直接攻撃するプランです。どれだけ駒損してもチェックメイトすれば勝ちなので、白は駒を捨ててチェックメイトを狙いに行くこともあります。

センターからの攻撃は、English Attackではあまり見られませんが、6. Bg5 Lineではよく見られます。

キングの周りの守り駒を除去する手として、{\bf g4-g5-g6!}が一つのテーマです。{\bf f3-f4-f5}もなくはないですが、あまり効果的ではないという印象です。一度7. f3と突いているのと、e4ポーンが浮くのがやや白にとっては怖いでしょうか。

ポーン・ストームは、相手のポーンが自分のポーンよりも少ない時に非常に効果的です。そのため白はキングサイドの黒ポーンを減らす手があれば好ましい成果を得られます。

{\bf Nd4xe6!}というテーマは、キングの守りを薄くする、展開が白のほうが早いためピースサクリファイスしてもピースアクティビティで十分代償がある、黒のセンターポーンを減らせる可能性がある(センターからの反撃を弱める)、という意味で効果的なテーマです。

{\bf Bf1-h3!}としてe6ポーンと、(ほとんどの場合)c8にいるルークを狙うテーマもあります。

白の白マスビショップは{\bf Bf1-d3}としてh7を狙うこともあります。

また別のサクリファイスとして、{\bf Nc3-d5!}というテーマもあります。黒にe6xd5とされても、eファイルが開き、黒の主張の一つであるセンターポーンが崩れ、白はピースを捨てただけのポジショナルな代償があるとされることが多いです。(Be3型よりも、Old Main Line 6. Bg5によくあらわれます)

キングサイドのルークはh1に置くほか、{\bf Rh1-e1}というテーマもあります。黒がキャスリングを遅らせているときに特に有効です。Nc3-d5と組み合わせられることもあります。

黒のクイーンサイドのポーン・ストームに対するディフェンスとしては、何もしない、{\bf Nc3-e2、a2-a3(-a4)、O-O-O-Kc1-Kb1}などがあります。b2-b3はだいたい失敗します。c2とc3が弱くなるのが、特にクイーンサイドにキャスリングした場合に致命的です。

面白い手としては{\bf Nc3(Nd4)xb5!}と、こちらのポーンを取る手もあります。黒の反撃が効かなくなるため、黒がゆっくりしている場合には効果がある場合があります。

\subsubsection{黒のテーマ}

キングサイドのディフェンス、センターからの反撃、クイーンサイドからの反撃があります。

キングサイドのディフェンスはNf6-d7!が良くある手です。キングサイドの焦土作戦で、攻めを空振りさせる意味です。

もちろん{\bf h7-h6}もあります。ここで白のg4-g5に対して{\bf h6-h5}と躱すこともできます。

白がf4-f5と攻めてきたときに{\bf e6-e5!}と指せると、白の攻めが止まることがあります。

センターからの反撃では、{\bf d6-d5!}がタイミングよく指せればほぼ互角でしょう。e6-e5を指した後か、指す前かは局面によります。

{\bf Nd7-e5}という動きもあります。e5にいるのはクイーンであることもあります。白はf4が付きづらいため(e4が落ちやすくなる)、e5のピースはしばらく居座れます。

白のe4ポーンが落ちると、黒のセンターが強力で白は相当苦しい戦いになります。

{\bf Nd7-c5}や{\bf Nd7-Nb6-Nc4}も面白い手です。c4のマスを黒が使うのが強力で、c4で白ビショップ-黒ナイトの交換になるとある程度オープンな局面で黒にビショップが残る形になるので終盤で有利になります。

クイーンサイドの攻撃としては何といっても{\bf Ra8-Rc8-Rxc3!}でしょう。これに対し白がbxc3しかなければエクスチェンジダウンでも黒が相当優勢です。

{\bf Qd8-c7}もあります。{\bf Qc7-a5}もあります。

{\bf b5-b4}もナイトを追い払う手としてよくあります。ただしb4後のNc3-Nd5!には注意するのと、c4マスを使えなくなることがあるため注意が必要です。

{\bf a6-a5}もなくはないです。この場合b5の地点が若干弱くなるため、注意が必要。

これを見ると、黒のクイーンサイドへのアタックはチェックメイトというよりもポジションを崩すことに主眼が置かれています。黒のアタックはa,bポーンで相手のa,b,cポーンを攻撃する、Minority Attackと呼ばれるアタックであり、このアタックのテーマは相手のポジションに弱点を作ることです。

これらのテーマを、現局面に当てはめて何が効果的かを考えながら組み合わせていくのがNajdorfの序盤-中盤の入り口といえると思います。


\subsection{6. Be3 e6 7. f3 b5型}
6. Be3 e6 7. f3のEnglish Attack型のフォーメーションに対して7... b5は"provocative"と言われることの多い手です。

このラインを黒で指すGMはVera Gonzalez, Van Wely, Gelfandなどです。黒はキャスリングを遅らせ、クイーンサイドにプレッシャーをかけるプランを取るラインです。
{\bf 1. e4 c5 2. Nf3 d6 3. d4 cxd4 4. Nxd4 Nf6 5. Nc3 a6 6. Be3 e6 7. f3 b5}
Emmsの本では8. Qd2と8. b4が主要なラインとして載っていました。

\def\fena{rnbqkb1r/5ppp/p2ppn2/1p6/3NP3/2N1BP2/PPP3PP/R2QKB1R w KQkq b6 0 8}
\begin{center}
\chessboard[setfen=\fena]

7... b5まで
\end{center}

\subsubsection{ 8. Qd2}

8. Qd2に対しては8... Nbd7!が好手とのことです。さらに9手目で2つに分かれます(9. O-O-Oと9. g4)。

\paragraph{8. Qd2 Nbd7 9. g4}
\mbox{}\newline

{\bf 8. Qd2 Nbd7 9. g4 Nb6 10. O-O-O Bb7}

黒はNc4でのフォークで、ナイトをどちらかのビショップと交換することを目標に指しています。

{\bf 11. Qf2}

白はb6のナイトにプレッシャーをかけます。この手ではなく、例えばBd3等であれば黒は11... Rc8として十分なようです。

{\bf 11... Nfd7}

ここは白の手が広いところですが、黒はRxc3を狙いに戦うことになります。一例として

{\bf 12. Bd3 Rc8 13. Nce2}

ほかの手だとRxc3を食らいます。

{\bf 13... Qc7}

Nc5-Na4という指し方を狙います。

{\bf 14. Kb1 d5!}

Najdorfでよく出てくるfreeing moveです。白からこれを取ると黒としては不満がない形になります。

{\bf 15. e5! Qxe5}

ポーンサクリファイスをすることで、黒のキングを攻撃にさらします。

この後は手が広いところです(16. Nxb5!?等も可能)が、黒はポーン得と強いセンターという主張点を手に入れたのでこの後はキャスリングを狙っていくのが理にかなっていると思います。

\paragraph{8. Qd2 Nbd7 9. O-O-O}
\mbox{}\newline
本当に微妙な違いですが、9... Nb6!?は似たようで違う形になります。

{\bf 9... Nb6!? 10. Qf2! Nfd7 11. f4! Bb7 12. f5! +-}

白がgポーンを突いていると、白のf4突きに対してNf6からのe4とg4ポーンへのアタックや、1手速いことによってビショップがb7にいるため黒からのb4突きでのe4のunderminingなどのタクティクスが黒に生まれますが、gポーンを突いていないとf4-f5突きが強烈です。

そのため、

{\bf 9... Bb7}

を先に指すことが必要です。ここで10. g4であれば10... Nb6として先のラインに戻せます。

多かれ少なかれ、白はKb1, g4などの手が必要なため、先のラインにトランスポーズできるでしょう。

白がディフェンスを重視するラインもあります。

{\bf 10. a3 Rc8 11. g4 Nb6 12. Nb3 Nfd7 13. Kb1 d5}

黒は無理に攻めず、センターを交換するだけで満足するのが良いと思います。

\subsubsection{8. g4}

8. g4に対しては8... Nbd7?は悪手です。9. g5!で白が良くなります。

{\bf 8. g4 h6!} (8... Nfd7!? 9. Qd2 Nb6 10. O-O-O N8d7 11. Ndxb5!(Shirov) +=)

\paragraph{8. g4 h6 9. Qd2}
\mbox{}\newline

{\bf 9... b4!}

8. g4 h6の形での黒のプランは基本的に、b4を突いてナイトをどかしてからe5, d5の順でポーンを突いていくことのようです。

{\bf 10. Nce2 e5 11. Nf5 d5 12. O-O-O Be6}

以下、黒は駒展開を進めれば悪くない形になります。


\paragraph{8. g4 h6 9. h4}

{\bf 9... b4!}

ポイントは同じです。

{\bf 10. Nce2 e5 11. Nb3 d5 12. Ng3 Be6}

11. Nf5にも11... Be6で十分です。

{\bf 13. Bd3 Nbd7 14. Qe2 a5}

これでクイーンサイドのスペースを確保して黒十分、というのがEmms本の主張ですが、黒のプランが見えづらい局面です。この局面でのプランニングは、今後の課題としたいと思います。

\subsection{7. g4!?}
Sicilianの5手目の黒の分岐(1. e4 c5 2. Nf3 d6 3. d4 Nxd4 4. Nxd4 Nf6 5. Nc3の後)で、5... e6に対して6. g4!がKeresの指した手で、Karpovが発展させた形で優秀である、という話を本章前半でしました。

これは、5... e6によってg4にビショップが効かなくなるので指せる手でした。

これを応用して、5... a6 6. Be3 e6に対して7. g4!?が可能かどうか、ということが5... a6, 6...e6型のポイントになります。

もし7. g4!?で白が良ければ、5... a6型Scheveningenの成立にも関わる重要な形です。

7. g4!?を指すGMは何といってもShirovです。(面白いことにSemi-Slavの7. g4!?という手にも、Shabalov-Shirov Gambitという名前が付いています)

では、見ていきましょう。

\subsubsection{7. g4!? e5}

7... e5がメインラインと言われていますが、非常に難しいラインです。もちろん、gポーンを取ろうとしている手です。黒がポーン得できるかが争点になります。

\def\fenb{rnbqkb1r/1p3ppp/p2p1n2/4p3/3NP1P1/2N1B3/PPP2P1P/R2QKB1R w KQkq - 0 8}
\begin{center}
\chessboard[setfen=\fenb]

7... e5まで
\end{center}

{\bf 8. Nf5!}

ナイトをf5に飛ぶことで、gポーンを取られることを防ぎます。

{\bf 8... g6}

ナイトに当てます。ナイトが動けばgポーンが取られるため、

{\bf 9. g5!}

白は突っ張って指すしかありません。

{\bf 9... gxf5! 10. exf5!}

10. exf6?は、10... f4!が好手で黒が良くなります。

{\bf 10... d5!}

10... Nfd7は11. Qh5!など。

{\bf 11. Qf3 d4!}

ポーンフォークが入りますが、

{\bf 12. O-O-O! Nbd7}

12. O-O-O!で切り返せます。この後は、Leko-Anand(2008)のゲームを追いましょう。

{\bf 13. Bc4 Qc7 14. Bxd4! (おそらくLekoによる、Sokolovの手14. Bb3の改良です) exd4 15. Rhe1+ Kd8 16. Rxd4 Bc5 17. Rdd1 Re8 18. gxf6 Rxe1 19. Rxe1 Nxf6 20. Rd1+ Bd7 21. Bxf7 Qxh2!? 22. Nd5 Rc8 23. Be6 Bxf2 24. c3 Rc7? 25. Nxf6! Qh6+ 26. Kb1 Qxf6 27. Qxf2 Ke8 28. Qg3 1-0}

\subsubsection{7. g4!? h5!?}

Emmsの本で推奨されているのはこの手です。g4マスに駒の効きを足すのは7... e5と同じですが、7... e5と違い一度突いたポーンをもう一度突く手ではありません。さらにEmmsによれば、白が1手Be3に手を使っているため、黒からのNg4という手が有力になるというのがこの手のもう一つのポイントです。


\def\fenc{rnbqkb1r/1p3pp1/p2ppn2/7p/3NP1P1/2N1B3/PPP2P1P/R2QKB1R w KQkq h6 0 8}
\begin{center}
\chessboard[setfen=\fenc]

7... h5まで
\end{center}

8. gxh5も調べる必要がある変化ですが、黒はh5ポーンを無理に取ろうとせず、9... b5!から駒展開を続けていけば自然なNajdorfらしい展開になります。よりクリティカルな8. g5についてみていきます。

{\bf 8. g5 Ng4 9. Bc1}

9. Bd2? Qb6! 10. f3 Nc6!で黒やや良しです。

{\bf 9... Qb6}

狙いはNc6と組み合わせて、f2を狙うことです。

{\bf 10. h3 Ne5 11. Be2 g6}

これで白のキングサイドに対する攻撃が難しくなり、黒は駒の展開を続けられます。この後はAlmasi-Judit Polgar(1996)戦に従います。

{\bf 11... Nbc6 12.Nb3 g6 13.Be3 Qc7 14.f4 Nd7 15.Qd2 b5 16.O-O-O Bb7 17.Rhf1 Rc8 18.Bd3 Be7 19.Kb1 O-O 20.Ne2 Nb6 21.f5 Ne5 22.Bxb6 Qxb6 23.Nbd4 Nxd3 24.cxd3 e5 25.Nf3 b4 26.Rc1 a5 27.Nh4 d5 28.Rxc8 Rxc8 29.Ng3 Ba6 30.fxg6 fxg6 31.Nf3 Qe6 32.Rg1 d4 33.Nh4 b3 34.Qxa5 Bxd3+ 35.Ka1 Qc6 0-1}


\subsection{Kasparov新手 8... Nfd7の動向}
\subsubsection{1. Kasparov新手 8... Nfd7}

Judit Polgar - Garry Kasparov, 2001, Cannes (Rapid)

{\bf 1. e4 c5 2. Nf3 d6 3. d4 cxd4 4. Nxd4 Nf6 5. Nc3 a6 6. Be3 e6 7. f3 b5 8. g4}

\def\fend{rnbqkb1r/5ppp/p2ppn2/1p6/3NP1P1/2N1BP2/PPP4P/R2QKB1R b KQkq g3 0 8}
\begin{center}
\chessboard[setfen=\fend]

8. g4まで
\end{center}

{\bf 8... Nfd7!?}

Kasparovはこの手に「!?」を付けています。(Garry Kasparov on Garry Kasparov, Vol.3) よくある手としては8... h6です。8... Nfd7はKasparovがAnandとの2000年のマッチで指した新手で、そのゲームはドローになっています。

{\bf 9. Qd2 Nb6 10. O-O-O}

白はすでに展開を完了していますが黒はナイトを展開しただけです。これだけ展開に差が付くと普通は白良しとしたものですが、白の駒に具体的な攻撃目標がないために黒はこれから展開を進めてもまだ間に合うという点でバランスが取れているかと思います。

なお、前述のKasparov-Anand戦では10. a4でした。

{\bf 10... N8d7 11. Qf2!?}

Kasparovはこの手に何もコメントを付けていませんが、後述するようにここでより良い手があると考えられます。

{\bf 11... Bb7 12. Bd3 Rc8 13. Nce2 Nc5}

まだ展開に差がありますが、黒が展開で追いついてきた印象もあります。

{\bf 14. Kb1 Nba4}

Najdorfで覚えておきたい手です。白のb3を誘い、クイーンサイドを弱めます。

{\bf 15. b3}

このあたりは、どのように指しても複雑な戦いになります(Kasparov)。

{\bf 15... Nxd3 16. cxd3 Nc5 17. Ng3}

コンピュータによれば、この辺りはすでに黒良しとのこと。あとは長いため、省略しますが、黒がリードを保って勝ちました。

{\bf 17... Be7 18. Qb2 b4 19. Nh5 Rg8 20. Ne2 g6 21. Nhf4 a5 22. d4 Nd7 23. d5 e5 24. Nd3 Ba6 25. Qd2 Bf6 26. Rc1 Bb5 27. g5 Bg7 28. Nb2 Ke7 29. f4 exf4 30. Bxf4 Qb6 31. Be3 Qa6 32. Nd4 Ne5 33. Rhd1 Bd7 34. Bf4 Rxc1+ 35. Rxc1 Rc8 36. Bxe5 Rxc1+ 37. Qxc1 Bxe5 38. Nc6+ Bxc6 39. dxc6 Qe2 40. c7 Bxb2
41. c8=Q Bxc1 42. Qb7+ Kf8 43. Kxc1 Qxh2 44. Qa8+ Kg7 45. Qxa5 Qf4+ 46. Kd1 Qxe4 47. Qd8 Qb1+ 48. Ke2 Qxa2+ 49. Kf1 Qa1+ 50. Kg2 Qe5 0-1}

さて、この手が成立するのであれば、8. g4に対しても8. Qd2に対しても黒は同じ陣形で戦えます。キングサイドに手をかけないNb6-Nd7-Bb7-Rc8型はEnglish Attackに対してかなり優秀な形であるため、白がこれを阻止するために早い8. g4を指していましたが、Kasparovの8... Nfd7!?が成立するのであればEnglish Attackの根幹にかかわる可能性もあります。

さて、8. g4 Nfd7に対する白の対策を出したのが2001年のShirovです。

\subsubsection{Shirovによる反駁 11. Ndxb5(13. Ndxb5)}

Alexei Shirov - Kiril Georgiev, 2002, FIDE Grand Prix

{\bf 1. e4 c5 2. Nf3 d6 3. d4 cxd4 4. Nxd4 Nf6 5. Nc3 a6 6. Be3 Ng4 7. Bc1 Nf6 8. f3 e6 9. Be3}

本筋とはあまり関係がないため詳述を避けますが、6... Ng4(Anti-English)に対して7. Bg5と出る手と7. Bc1に引く手があります。後者は、このゲームのように黒がNf6に戻るため、白が手を変えない限り黒はドローにするという意図を持っています。

{\bf 9... b5 10. g4 Nfd7}

手数は2手多いですが、8... Nfd7と同じ形になりました。

{\bf 11. Qd2 Nb6 12. O-O-O N8d7}

\def\fene{r1bqkb1r/3n1ppp/pn1pp3/1p6/3NP1P1/2N1BP2/PPPQ3P/2KR1B1R w kq - 5 13}
\begin{center}
\chessboard[setfen=\fene]

12. N8d7まで
\end{center}

さて、Shirov新手は。

{\bf 13. Ndxb5!}

いきなりのサクリファイスですが、黒の展開の遅れを突いた手です。この後はほぼ必然の進行です。

{\bf 13... axb5 14. Nxb5! Ba6 (14... d5? 15. Qc3!! +-) 15. Nxd6 Bxd6 16. Qxd6}

白はピースの代わりに3ポーンを得ており、主導権を握っており、次に17. Bxb6のスレットもあります。黒が白の主導権を弱めようとピースを返すと白はクイーンサイドにパスポーンができます。実戦的には白が相当勝ちやすいと思います。

{\bf 16... Nc4 17. Bxc4 Bxc4 18. Qd4 Be2 19. Qxg7 Qf6?}

ここで19... Rf8がのちの改良手ですが、白はその前に18. a3!と改良手があり、この局面は白良しと考えられています。

{\bf 20. Qxf6 Nxf6 21. Rde1 Bxf3 22. Rhf1 Bxg4 23. Bd4 Rxa2 24. Kb1 Ra8 25. Bxf6 Rg8 26. Re3 Kd7 27. b3 Kc6 28. Rf2 Bh5 29. Rd2 Rg1+ 30. Kb2 Rd1 31. Rxd1 Bxd1 32. c4 Rg8 33. Rg3 Rb8 34. e5 Rb7 35. Rd3 Be2 36. Rd6+ Kc7 37. Bd8+ Kb8 38. Bb6 Re7 39. Rd8+ Kb7 40. Bc5 Rc7 41. Bd6 Rc8 42. Rxc8 Kxc8 43. Be7 Kd7 44. Bf6 Kc6 45. Kc3 h5 46. b4 1-0}

\subsubsection{Ndxb5の成立条件}
さて、Shirovのこの手(Ndxb5)は8. g4 Nfd7型でのみ成立する手でしょうか。

8. Qd2 Nbd7 9. g4 Nb6型を見ていきましょう。

{\bf 1. e4 c5 2. Nf3 d6 3. d4 cxd4 4. Nxd4 Nf6 5. Nc3 a6 6. Be3 e6 7. f3 b5 8. Qd2 Nbd7 9. g4 Nb6 10. O-O-O Bb7!}

10. N8d7であれば同じ形になりますが、黒はここで10... Bb7!とできます。白が11. Qf2としてb6のナイトを狙ってきたときに11... N8d7とするのが正しい順序です。

ここで11. Ndxb5とするとどうなるか。

{\bf 11. Ndxb5?! axb5 12. Nxb5 d5! -+}

d5地点に効きが多いのと、Rc8が可能(13. Qc3? Rc8!)であるため、dポーンが取られることはありません。Nb5-(Qc3)-Nc7を含みにして、Nxd6として3ポーンを取ることが可能であることが、Ndxb5が成立する条件になると考えられます。

微妙な手順前後により、手が成立するかどうかが変わる例でした。

\subsection{10... Bb7の成立可否}
{\bf 1. e4 c5 2. Nf3 d6 3. d4 cxd4 4. Nxd4 Nf6 5. Nc3 a6 6. Be3 e6 7. f3 b5 8. g4 Nfd7 9. Qd2 Nb6 10. O-O-O}

この変化は、手順が重要であることはこれまで述べてきたとおりです。8. Qd2から9. g4と8. g4から9. Qd2で、全く異なった局面になります。
そして、10... N8d7が成立しないことは前の項で述べた通りです。

しかしながら、ここで10... Bb7が成立すれば、第5回でも書いたように、「8. Qd2から9. g4」でも「8. g4から9. Qd2」でも黒は同じ局面で戦うことができます。

そんなことが可能だろうか、というのが今回のテーマです。

{\bf 10... Bb7}
いかにも、黒は無理をしている陣形です。白がもし普通に11. Qf2と続けるのであれば11... N8d7!で、黒は満足です。

ここで白には次の手があります。

{\bf 11. Nb3!}

これが素晴らしい手で、次に12. Na5!としてb7のビショップ取りを狙います。黒は展開で遅れているうえにビショップペアを失うと、黒の戦略目標の一つであった「...d5からセンターを開く」ことが、非常に危険になります。1999年、Bologanの手です。

ここで初志貫徹の11... N8d7が2002年のAnand-Ponomariov戦でPonomariovによって指された手です。ただしこのゲームは、12. Na5からAnandが快勝しました。

ここで、2つの手段があります。

\subsubsection{ 11... Nc6}

{\bf 11... Nc6}

12. Na5を防ぐためにはこうするしかありません。しかし、この時にb6のナイトの守りに効いていないことに目を付けて、

{\bf 12. Qf2}

こう指すことができます。

{\bf 12... Nd7}

この局面は白良しです。黒c6のナイトに役割がありません。

\subsubsection{11... b4!}

あまり指されていませんがよりクリティカルなのは、11... b4!です。Hikaru Nakamuraの手です。

白にもここでいくつかの手段があります。
\begin{enumerate}
\item 12. Bxb6!?
\item 12. Nb1
\end{enumerate}

12. Bxb6!? は、12... bxc3 13. Qe3 cxb2+ 14. Kb1 Qc8のように続き、黒がb,cファイルから圧力をかけられるでしょう。特にbファイルが危険です。
12. Nb1がメインムーブです。

{\bf 12. Nb1 Nc6 13. Qf2}

先ほどと同様に進みますが、

{\bf 13... Na4!}

ここでこの手が可能です。こうなるとc6のナイトもb4のポーンを支えており、悪い形ではないです。

さらに、このナイトはなかなか取られない(白b3を指すことが難しい)上に、白キングにプレッシャーをかけている形になります。

まだこれからの勝負でしょう。

結論としては、
\begin{enumerate}
\item 8. g4から9. Qd2の手順に対して10... Bb7として、Nb6-Nd7-Bb7-Rc8型に組もうとする手に対しては11. Nb3!があり、Na5を見せられるためNb6-Nd7-Bb7-Rc8型には組めない。
\item ただし、11... b4!があり、黒も十分戦うことは可能な局面である。
\end{enumerate}
ということになります。


\subsection{実戦例}
最後に、最近のゲームを2局載せ、本章を締めたいと思います。
\subsubsection{Game 1: Morozevich A. - Vachier-Lagrave, M. (2009)}

まずは、Mr.Najdorfとの呼び声高いフランスのGM、Maxime Vachier-Lagrave(MVL)のゲームより。

MVLは1. e4に対してほぼいつも1... c5で返し、さらにオープンシシリアンに対してはNajdorfを指すことで有名です。結果として、MVLの全ゲームのうち1割弱がNajdorfです。

2009年、白Morozevichに対するMVLのゲームより。Biel GMトーナメントの8Rという大一番です。

{\bf 1. e4 c5 2. Nf3 d6 3. d4 cxd4 4. Nxd4 Nf6 5. Nc3 a6 6. f3 e6 7. Be3 b5 8. Qd2 Nbd7 9. g4}

\def\fenf{r1bqkb1r/3n1ppp/p2ppn2/1p6/3NP1P1/2N1BP2/PPPQ3P/R3KB1R b KQkq - 0 9}
\begin{center}
\chessboard[setfen=\fenf]

9. g4まで
\end{center}

{\bf 9... h6}

MVLは9... Nb6も指しています。

{\bf 10. O-O-O b4 11. Nce2 Qc7 12. h4 d5}

タイミングよく...d5を突け、黒悪くありません。

{\bf 13. Nf4 e5 14. Nfe6!}

並べてみるとわかりますが、何もないマスにナイトをサクリファイスしています。

{\bf 14... fxe6 15. Nxe6 Qa5 16. exd5!}

...Qxa2を受けない!これでe6にナイトを固定します。

{\bf 16... Qxa2 17. Qd3 Kf7 18. g5 Nxd5!}

これでe6のナイトを取り切ってしまえば白には駒損だけが残りますが……

{\bf 19. Bh3 Nxe3 20. Nd8+!}

2ピースダウンで戦います。

{\bf 20... Ke7 21. Nc6+ Kf7 22. g6+ Kg8 23. Qxe3}

\def\feng{r1b2bkr/3n2p1/p1N3Pp/4p3/1p5P/4QP1B/qPP5/2KR3R b - - 0 23}
\begin{center}
\chessboard[setfen=\feng]

23. Qxe3まで
\end{center}

霧が晴れました。黒はピースアップですが、両方のルークが働いておらず、クイーンも変な位置にいて、キングも安全ではありません。加えてdファイルは白が支配しており、ピース損の代償は十分ある局面でしょう。

{\bf 23...Bc5 24. Qe4 Nf8 25. Rd8 Bb7 26. Rxa8 Bxa8 27. h5 Rh7!!}

キングの逃げ道を塞いでいるルークより、g6のポーンのほうが価値が高いという判断はすごいと思います。

確かに今のままだと、黒のルークは動けず、ナイトもバックランクメイトを防ぐために動きが制限されています。加えて白マスを守らないとQc4+等からのメイトがあるため、クイーンとビショップの動きは気を付けないといけません。結果として、黒が自由に使えるのは黒マスビショップのみとなります。

それに比べればルークを1つ取られてエクスチェンジダウンになってもダブルビショップで戦えるという判断でしょう。

{\bf 28. Re1!}

白も取りません。

{\bf 28... Bxc6 29. Qxc6 Bd4 30. Kd2 Qxb2 31. Qc4+ Kh8! 32. Kd3 a5}

黒はここから、白の黒マスの弱さにつけ込んでいきます。31... Kh8!でh7取りがチェックにならないのも大きなポイントで、h7を取るとその瞬間に白がチェックメイトされる、という筋もいくつも出てきます。

{\bf 33. Qc8 Qa3+ 34. Ke4 b3 35. cxb3 a4 36. Rb1 Qb4 37. Qc4 Qb7+ 38. Qd5 Qb4 39. Qc4 Qd2 40. Bg4 a3 41. Qf7 Qc2+ 42. Kd5 Qc5+ 43. Ke4 a2 44. Rc1 a1=Q 45. Rxc5 Bxc5}

クイーンが世代交代しました。f8のナイトの位置が素晴らしく、メイトスレットがあります。

{\bf 46. Qd5 Qe1+ 47. Kd3 Qd1+ 48. Kc4 Qxd5+ 49. Kxd5 Ba3 50. Bf5 Kg8 51. Kxe5 Rh8 52. Kd5 Nh7
53. gxh7+ Kf7 54. Bg6+ Kf6 55. f4 Bc1 56. f5 Bd2 57. Kd6 Be1 58. Kd7 Bb4 59. Kc7 Ke5 60. Kd7 Ba3 61. Kc6 Kd4 62. Kc7 Kc3 63. Kd7 Kb4 64. Kd6 Kxb3+ 65. Kd5 Bb2 66. Kd6 Bf6 67. Kc5 Kc3 68. Kd6 Kd4 69. Kc6 Rd8 70. Kb6 Kd5 71. Kc7 Kc5 72. Bf7 g5 73. fxg6 Rd6 74. Be8 Be5 75. Kb7 Rb6+ 76. Kc8 Kd6 0-1}

\subsubsection{Game 2: Leko, P. - Shirov, A. (2012)}

さて、MVLはEnglish Attack(6... Be3)に対しては6... Ng4, 6... e5, 6...e6どれも指します。ただし、2013年以降はもっぱら6...e6をやめ、6...Ng4か6...e5に専念しているようです。

調べてみると、どうやら6...e6の、8. Qd2 Nbd7 9. g4 Nb6のラインで、現在では10. O-O-Oではなく10. a4とする手が強力と見られているようです。黒がキャスリングを遅らせるのに対抗して白もキャスリングを遅らせ、場合によってはO-Oまで見せながら、黒のクイーンサイドの攻勢を受け流します。

私が参考にしているEmms本が出た後(2004年以降)に流行りだしたラインなので、載っていないのも納得ですが、このラインについては改めて検証する必要があるでしょう。

黒も勝てないラインではなく、Shirovが黒番を持って勝っているゲームがあるので、それをご紹介します。

{\bf 1. e4 c5 2. Nf3 d6 3. d4 cxd4 4. Nxd4 Nf6 5. Nc3 a6 6. Be3 e6 7. f3 b5 8. Qd2
Nbd7 9. g4 Nb6 10. a4}

これが10. O-O-Oに代わって増えている手です。キャスリングして黒のアタックを正面から受けるよりも、いったん黒のクイーンサイドの動きを制限したうえでセンターから反撃するというプランです。黒は、次の手にあるように...Nc4としながらbファイルを開けて戦うようです。

{\bf 10... Nc4 11. Bxc4 bxc4 12. a5 Bb7 13. Na4 d5}

c4のポーンを支えます。この形はbファイルが開くので、ルークはbファイルに回します。

{\bf 14. g5 Nd7 15. O-O-O}

やや危険なようにも見えます。白には驚くことに、O-O!というプランもあります。

{\bf 15... dxe4 16. f4 Rb8 17. Qc3 Qe7 18. Nf5!}

Najdorfの白番はナイトサクリファイスで主導権を白が握るゲームが多いですね。

{\bf 18... exf5 19. Rxd7 Qxd7 20. Qe5+ Be7 21. Qxb8+ Qc8 22. Qe5 f6 23. gxf6 gxf6 24. Qd4 Qc6 25. Nc5 Bxc5 26. Qxc5 Qxc5 27. Bxc5 Rg8 28. Rg1 Rg4 29. Rxg4 fxg4}

異色ビショップエンディングになりましたが、ここからのShirovの差し回しは参考になります。有名なTopalov-Shirov(1998)を思わせる指し回しです。黒はgファイルにパスポーンを作れることと、eファイルにパスポーンがあることを使って戦いますが、異色ビショップであるため、なるべく離れたファイルにパスポーンを作りたい形です。

{\bf 30. Kd2 Kd7 31. Bf2 Kc6 32. Ke3 Kd5 33. Bh4 f5 34. Bf6 Bc6 35. b4 cxb3 36. cxb3 h5}

...h4から...g3を見せることで、キングとビショップの動きを制限します。加えて黒キングはクイーンサイドに向かい、弱いbポーンを狙いに行きます。白ビショップはbポーンを守り、黒キングの侵入を防ぐ必要があるうえに、...h4を止めないといけないため、両方の仕事ができるマス(e7)にい続けなければいけません。

加えて白キングは、あまりクイーンサイドに寄りすぎると黒から...e3!があります。

また、b4と突いてしまうと黒キングがクイーンサイドの白マスから侵入されるのを防ぐにはクイーンサイドに白キングが行くしかなく、キングサイドにコネクテッドパスポーンを作られて負けるため、b4を突くと黒キングが侵入できます。

白はいずれ、指せる手がなくなります。

{\bf 37. Be7 Be8 38. Kd2 Kd4 39. Bf6+ Kc5 40. Kc3 Bb5 41. Be7+ Kd5 42. Kd2 Bd3 43. Kc3 Bf1 44. Kd2 Kd4 45. Bf6+ Kc5 46. Be7+ Kb5 47. b4 Kc4 48. Ke1 Bd3 49. Kd2 Kb3 50. Ke3 Kc3}

\def\fenh{8/4B3/p7/P4p1p/1P2pPp1/2kbK3/7P/8 w - - 0 51}
\begin{center}
\chessboard[setfen=\fenh]

50... Kc3まで
\end{center}

白はツークツワンクになりました。

{\bf 51. Bh4 Kxb4}

51. Bf8等でb4のポーンを守るのも、51... h4 52. b5 g3 53. h3 Bxb5等で黒勝ちです。

{\bf 52. Be1+ Kc5 53. Bh4 Bb5 54. Be1 Bd7 55. Bh4 Kb5 56. Be1 h4!}

これで、当初の目的であった「離れたパスポーン(aファイル、eファイル)」を作れました。

{\bf 57. Bxh4 Kxa5 58. Kd4 Kb5 59. Be7 a5 60. Kc3 Be6 61. Kd4 Bc4 62. Bh4 Bf1 63. Be7 Ka4 64. Kc3 e3 65. Bh4 Ka3 66. Kc2 a4 67. Be7+ Ka2 68. Bc5 e2 69. Bb4 a3 0-1}