\section{French, Tarrach, Closed, Korchnoi Gambit}




\subsection{序論}
French Defence(1. e4 e6)も、1. e4に対する対策として人気があるオープニングです。
特に、2019年現在日本のプレーヤーの間では非常に流行している印象です。手堅い形を作れること、それでいながらポジショナルにもタクティカルにも黒から手を作っていけることなどが人気の理由でしょうか。

さて、ここで紹介するKorchnoi GambitはFrench Defenceに対して白番が採用する定跡です。名前の由来であるGM Korchnoiは黒番でFrenchを採用していたことで有名です。世界選手権にも2度登場した強豪中の強豪です。French Defenceの大家が見せるFrench破りを紹介します。

{\bf 1.e4 e5 2. d4 d5 3. Nd2}

Tarrasch Defenceです。黒の対応としては3... c5 (Tarrasch Open)と3... Nf6 (Tarrasch Closed)に分かれます。個人的な印象ですがClosedのほうがFrenchっぽいポーンストラクチャーになるのでよく見る気がします。

{\bf 3... Nf6 4. e5 Nfd7 5. Bd3 c5 6. c3 Nc6}

このあたりは定跡です。Closed TarraschはBd3のラインで使います。

{\bf 7. Ngf3!?}

\def\fena{r1bqkb1r/pp1n1ppp/2n1p3/2ppP3/3P4/2PB1N2/PP1N1PPP/R1BQK2R w KQkq - 0 1}
\begin{center}
\chessboard[setfen=\fena]

\end{center}

この手がKorchnoi Gambitに入るために必要な手です。Main Lineは7. Ne2です。7. Ne2 cxd4 8. cxd4 f6 9. Nf4!?というラインも激しいことで有名です。

 一見自然な手ですが、黒には次の手があります。

{\bf 7... Qb6}

これでポーンダウンを避けるためには白は8. Qa4か8. dxc5という手しかなく、主導権を黒に握られることになりそうです。そのため、以下のようにポーンを捨てます。

{\bf 8. O-O cxd4 9. cxd4 Nxd4 10. Nxd4 Qxd4}

一見白が大変まずそうな局面ですが、白には次の手があります。

{\bf 11. Nf3 Qb6 12. Qa4!}

11. Nf3で、e5のポーンとd3のビショップが同時に守れます。黒クイーンの退却に対し、12. Qa4!と指します。このあたりがKorchnoi Gambitの基本形だと思います。

\def\fenb{r1b1kb1r/pp1n1ppp/1q2p3/3pP3/Q7/3B1N2/PP3PPP/R1B2RK1 w kq - 0 1}
\begin{center}
\chessboard[setfen=\fenb]

\end{center}

この局面、黒の利点は明確で、ポーン、それもセンターポーンが白より1つ多いです。このまま駒交換が進めばdポーンもプロテクテッドパスポーンになりそうです。これは中盤-終盤では明確なメリットとなります。

一方欠点としては駒展開が若干遅れていることがありそうです。12... Nc5とは指せませんね。

白の利点・欠点は黒の逆で、駒展開は進んでいるが駒損しています。

この後、白の方針としてはQg4からキングサイドアタックを狙います。キングサイドアタックは以下の戦略に基づくはずです。
\begin{itemize}
\item 黒は黒マスビショップを展開しないといけない。
\item 黒が黒マスビショップを展開すると、g7ポーンが浮くため攻撃対象になる。
\item 白のクイーンがg4にいても、黒はNf6とは飛べず、ポーンチェーンで白マスビショップのラインが閉じているためクイーンに対するアタックを受けない。
\item Frenchの常套手段である、黒Pf6からセンターポーンを清算するラインはキングの斜めのラインが開くため、このラインの白Qg4の後では成立しないことが多い。
\item f7ポーンをプロテクトすることが難しい。ナイト・白マスビショップではすぐに守れず、当然黒マスビショップでも守れない。
\item cファイルが素通しのため、キングサイドアタックを受けた際にクイーンサイドに逃げ込めなくなることがある(Rac1が決め手になることがある)
\end{itemize}

そのため、黒のディフェンスは、白Qg4を防ぐことを主眼に考える必要があるはずです。

データベースを見ると、主に黒のディフェンスは3通りあるようです。
\begin{itemize}
\item {\bf 12... Bc5}
\item {\bf 12... Be7}
\item {\bf 12... Qb4}
\end{itemize}
Main Lineは12... Qb4で、有力なSide Lineは12... Be7らしいですが、まず12... Bc5を見てみようと思います。


\subsection{12... Bc5の変化}
一見自然な12... Bc5ですが黒は正確に受けないといけなくなります。変化を見てみます。

12... Bc5 に対しては、13. Qg4! が相当大変な手です。受け方は主に2通りで、13... Kf8と13... g6があります。

\subsubsection{12... Bc5 13. Qg4! Kf8}
13... Kf8に対しては、14.Bd2!?が面白い手。2つめのポーンを黒に献上する手ですが、14... Qxb2?!には15. Qf4!として次のNg5が破壊的です。例を挙げると15. Qf4 Qb6 16. Ng5 Nxe5 17. Qxe5 f6 18. Qg3 fxg5 19. Bxg5 Qd6 20. Be3(d4を突かせてBe4を作る)のような感じで攻めが続きます。

キングサイドを守る手、例えば14... Be7等であれば15. b4としてc5にピースを置かせないようにしてからルークをcファイルに回してcファイルを抑えてしまえば互角-やや指しやすい形勢と思います。

\subsubsection{12... Bc5 13. Qg4! g6}
13... g6に対しては、黒マスの弱点を突く14. Bh6!があります。やはり14... Qxb2には15. Rab1!として、15... Qxa2?!と3つめのポーンを取られたら16. Ng5とします。キングサイドを守る16... Be7には17. Bb5として攻撃が続きます。黒ナイトがピンされているのでNxe6-Qxe6が効きやすいのがポイント。

15... Qa3のほうがよく、16. Bb5 a6 17. Bxd7 Bxd7 18. Rxb7となります。

14... Bf8が本線ですが、15. Qf4 Qb4 16. Qc1 Qc5 17. Bxf8 Qxc1 18. Raxc1 Kxf8 19. Rc7として駒の効率で十分ポーン損の代償は取れているでしょう。

\subsection{12... Be7の変化}
主観ではこの変化が実戦で一番よく出会うように感じます。ここでも13. Qg4!は有力だと思っています。

基本的には13... g6 14. h4と進むことが多いです。この後白はキングサイドアタックを仕掛けていき、黒はそれを受けるという展開になります。
ちなみに黒としては、キングサイドアタックを避けるために14... Nc5 15. Bc2 Bd7 16. Rd1 O-O-Oとするのは若干急ぎすぎで、白はRb1-Be3-Qf4-Ng5のようにしてキングサイドでポーンを取り返せます。

13... g6 14. h4のあと、黒としてのプランはいくつかあります。
\begin{enumerate}
\item キングサイドにキャスリングし、正面から白の攻めを受け切る。
\item クイーンサイドにキャスリングし、キングサイドは軽く流す。
\item クイーンを交換して白のアタックを緩和する。
\end{enumerate}
1. は例えば14... O-O?であれば15. h5! Qb4 16. Qg3等で白十分。

2.を実現するためには白マスビショップを動かす必要があり、そのためにはナイトを動かす必要があります。そのため14... Nf8と14... Nc5等が考えられます。

3.は14... Qb4です。

\subsubsection{12... Be7 13. Qg4 g6 14. h4 Nf8}
黒マスを弱くするやや危険な手で、15. Bg5!と黒のグッドビショップを交換しに行く手があります。この手に対して15... Bc5?は16. Rac1 h6 17. Bf6 Rg8 18. Qf4!とします。ここで黒がポーンを守る18... h5?に対しては、19. Rxc5!!が成立します。19... Qxc5 20. Rc1 Qb6 21. Ng5!として次のNxf7からBd8(ディスカバードチェック)、Bxb6を狙っていけば白はっきり良し。

そのため、18... Bd7くらいですが、19. Qxh6として白十分でしょう。

15. Bg5に対して15... Qb4であれば16. Qxb4 Bxb4 17. Bf6 Rg8 18. Rac1として、cファイルを支配しておけば白十分。

\subsubsection{12... Be7 13. Qg4 g6 14. h4 Nc5}
この手は自然な手に見えますが、黒としてはしばらくQb4を指せなくなるというデメリットがあります。Korchnoi Gambitにおいて黒のQb4は白のプレッシャーを解消する最良の手段なので、ひとつプレッシャーを解消するオプションを無くしてしまったということになります。

例えば、この後15. Bc2 Bd7 16. Rd1 O-O-O!? 17. Rb1 Bc6? 18. b4!のように進み、b4のマスをしっかりと抑えられます。17... Bb5!?が正しく、18. b4! Na4 19. Be3のように進みます。黒クイーンとキングがアタックを受けやすく、実践的には危険でしょう。

\subsubsection{12... Be7 13. Qg4 g6 14. h4 Qb4}
Korchnoi Gambitのrefutationとしては、やはりどこかのタイミングで黒Qb4を指すことに軍配が上がるでしょう。(しかし、知らないと指せない手でもあります。)この後は例えば15. Qg3 Nc5(コンピュータは15. Qxb4 Bxb4を推奨しますが、白の一貫したプランが見えません。) 16. Bd2!?のように進むでしょう(ちなみにうっかり16... Qxb2は17. Rfb1! Qa3?? 18. Bb5+! Bd7 19. Bb4 Ne4 20. Bxd7+! Kxd7 21. Bxa3 Nxg3 22. Rxb7+ Kc6 23. Rxe7となります。)


\subsection{12... Qb4の変化}
これがメインラインとしてMCO(Modern Chess Openings)にも載っているラインです。以下13. Qc2 Qc5 14. Bxh7のように進み、白はポーンを取り返します。このラインで黒はイコアライズに成功するというのが一般的な評価です。なので、13. Qc2 Qc5 14. Qe2!のように白は工夫する必要があります。

そのあとは例えば14... Qb6 15. a3 Nc5 16. Be3 Qd8 17. Bxh7!など(17... Rxh7 18. Bxc5 Bxc5 19. Qb5+ Bd7 20. Qxc5となり、評価が難しい局面ですがポーンが多く残っていてマイナーピースがバッドビショップ対ナイトなので、QとRをさばいてしまってマイナーピースだけの終盤にすれば白が面白く指せるはずです)。

総じてKorchnoi Gambitは白が全体的に強い圧力を盤面全体にかけられる変化であり、研究しがいのある変化だと思います。

\subsection{Anti-Korchnoi Gambit}
Korchnoi Gambitは正しく指せば黒悪くはならないのですが、キャスリングが遅れる、全体的にスペースアドバンテージがない等、黒にも嫌な部分はあります。

そのため黒からKorchnoi Gambitを避ける手も定跡化されています。d4ポーンを取らず、キングサイドのポーンを突くことでキングサイドアタックを緩和するというのが主なプランになります。

1. e4 e6 2. d4 d5 3. Nd2 Nf6 4. e5 Nfd7 5. Bd3 c5 6. c3 Nc6 7. Ngf3!?でKorchnoi Gambitとの境目になります。ここで7... Qb6 8. O-O cxd4とすればKorchnoi Gambitのメインラインですが、dポーンを取りにいかないことも黒は可能です。

\def\fenc{r1bqkb1r/pp1n1ppp/2n1p3/2ppP3/3P4/2PB1N2/PP1N1PPP/R1BQK2R w KQkq - 0 1}
\begin{center}
\chessboard[setfen=\fenc]

7. Ngf3!?まで
\end{center}

\subsubsection{7... cxd4 8. cxd4 f6}
Closed Tarraschの7. Ne2 メインラインと同じように進める場合。ちなみに、Mikhail Talが9. Ng5!? fxg5 10. Qh5+ g6 11. Bxg6+!?とやっていますが、成立しているかは微妙なところ。

実際、メインラインと同様に進めるとどうなるのか。以下で見ていきます。なお、本稿においては7. Ngf3から派生する局面を「7. Ngf3型」、7. Ne2(Main Line)から派生する局面を「7. Ne2型」と呼びます。

7... cxd4 8. cxd4 f6 9. exf6 Nxf6 10.O-O! Bd6で下図。

\def\fend{r1bqk2r/pp4pp/2nbpn2/3p4/3P4/3B1N2/PP1N1PPP/R1BQ1RK1 w kq - 2 11}
\begin{center}
\chessboard[setfen=\fend]

\end{center}

Tarrasch Main Lineでは10. Nf3と指すところ、すでにf3にナイトがいるので10.O-O!とできます。

ここで、Tarrasch Main Lineと比較してみましょう。少し長いですが、11. O-Oまで下のとおりです。

{\bf 1.e4 e6 2.d4 d5 3.Nd2 Nf6 4.e5 Nfd7 5.Bd3 c5 6.c3 Nc6 7.Ne2 cxd4 8.cxd4 f6 9.exf6 Nxf6 10.Nf3 Bd6 11.O-O}

\def\fene{r1bqk2r/pp4pp/2nbpn2/3p4/3P4/3B1N2/PP2NPPP/R1BQ1RK1 b kq - 3 11}
\begin{center}
\chessboard[setfen=\fene]

\end{center}

上(7. Ngf3)と下(7. Ne2)を比較すると、どうでしょうか。似た局面ですが、違いを理解してそれぞれ個別のプランに結び付けることが重要です。

違いは白ナイトの位置と手番ですね。

7. Ngf3型はNd2、7. Ne2型はNe2にナイトがいます。そのため、7. Ngf3型ではeファイルがハーフオープンになっていて、かつルークをすぐに敵陣に直射させられる陣形になっています。

加えて、手番は7. Ngf3型が白、7. Ne2型が黒になっています。このことを考えると、7. Ngf3型のほうが白は攻勢を取りやすいと言えるでしょう。

次に黒の陣形について考えます。黒の陣形としては、e6ポーンが弱くなっているが、ハーフオープンになったfファイルにルークを配備しつつキャスリングする手が可能です。そのため、黒も十分反撃はできるといえるでしょう。

上記の検討から、7. Ngf3型では、11. Re1と指し、黒のfファイルからの反撃が来る前にeポーンにプレッシャーをかけていくことが可能です。例えばそのあとはNb3, Qe2のような陣形を作り、eポーンを狙って指せば白指しやすいでしょう。

逆に黒は、7. Ngf3を相手にメインラインと同じ指し方をする際には、同じプランで指してはいけないということがわかると思います。

\subsubsection{7... Be7 8. O-O g5!}
割と意外性の高いプランです。というのは、ここで黒は、通常のFrenchのラインではあまり採用されることがない、キングサイドにプレッシャーをかけるプランを選択しているからです。このプランは、全体的に非常な乱戦になります。GM Nakamura、GM Volkovが黒を持ってしばしば指すラインでもあります。

さて、この手に対する対策はどうするのがいいのでしょうか。

{\bf 7... Be7 8. O-O g5! 9. dxc5}

Frenchでは珍しく、白からcポーンを取ります。白のセンターポーンは崩壊しますが、代わりに盤面全体にプレッシャーをかけることでバランスを取ります。

{\bf 9... Ndxe5 10. Nxe5 Nxe5 11. Nb3!}

この場所に居座ったナイトが容易に取られないため、c5のポーンも守られるというのが主張です。

{\bf 11... Nxd3 12. Qxd3}

\def\fenf{r1bqk2r/pp2bp1p/4p3/2Pp2p1/8/1NPQ4/PP3PPP/R1B2RK1 b kq - 0 12}
\begin{center}
\chessboard[setfen=\fenf]

\end{center}

この後は白は盤面全体の黒マスを抑えるためにf4突きを狙い、黒はそれを妨害する、というプランに沿って手順が進みます。なので12... e5!が良い手。黒がダブルビショップを持っているのに対して白はナイト+ビショップを持っているため、白としてはなるべくクローズな局面にすることが肝要です。

French Tarraschの7. Ngf3型は、鉱脈も多く残っている面白い戦型だと思います。French対策として、十分考慮に値する定跡ではないでしょうか。