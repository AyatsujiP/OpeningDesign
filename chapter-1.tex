\section{Caro-Kann, Advance, Tal Variation}

\def\fena{rn1qkbnr/pp2pppp/2p5/3pPb2/3P3P/8/PPP2PP1/RNBQKBNR b KQkq - 0 4}
\def\fenb{rn1qkbnr/pp1b1pp1/2p1p2p/3pP2P/3P1PP1/8/PPP5/RNBQKBNR b KQkq - 0 7}
\def\fenc{rn1qkbnr/pp3ppp/2p1p1b1/3pP2P/3P2P1/5P2/PPP5/RNBQKBNR b KQkq - 0 7}
\def\fend{rn1qkbnr/pp2ppp1/2p4p/3pPb2/3P3P/8/PPP2PP1/RNBQKBNR w KQkq - 0 5}
\def\fene{rn1qkbnr/pp2ppp1/2p4p/3pPb2/3P2PP/8/PPP2P2/RNBQKBNR b KQkq g3 0 5}
\def\fenf{rn2k2r/pp1bnpp1/4p2p/2qpP3/6PP/2P2N2/PP3P2/R1BQKB1R w KQkq - 0 12}


\subsection{序論}
Caro-Kann Defense(1. e4 c6)は、非常にソリッドなオープニングとして知られています。1... c6は駒展開には影響しない手ですが、その代わりd5の地点を非常にしっかりと抑えることができます。また、French Defense(1... e6)と違い、c8のビショップの展開を妨げていないため、c8のビショップをf5やg4に出してから...e6と指すことで、c8のビショップがポーンの内側に閉じ込められることを防ぎます。

Caro-Kannに対する白の手段はいくつかありますが、近年流行しているのがAdvance Variation(1. e4 c6 2. d4 d5 3. e5)です。センターからキングサイドにかけてのスペースの広さを主張する手です。これに対しては黒は3... Bf5と、ビショップを出すのが一般的です。この手に対して、4. h4!?とする手をTal Variationと呼びます。

\begin{center}
\chessboard[setfen=\fena]

4. h4!?まで
\end{center}

\begin{bfenumerate}
\setcounter{enumi}{0}
\item {e4 c6}
\item {d4 d5}
\item {e5 Bf5}
\item {h4!?}
\end{bfenumerate}

このオープニングは、Talが1961年の世界選手権で、Botvinnikに対して連採したことで知られているオープニングです。その時にはあまり効果を挙げることはなく終わりました。しかし、それと同時に、また別の世界選手権でも非常に大きな役割を果たしたオープニングでもあります。
2004年のKramnik対Lekoの世界選手権、最終14R、勝たなければ世界チャンピオンの称号を失うゲームでKramnikが選んだのが、このオープニングでした。彼はこのゲームで6手目に新手を指し、そのまま勝利しています。

\subsection{序盤の考え方}
まず、4. h4はどのような狙いを持った手かを考えます。

\begin{center}
\chessboard[setfen=\fena]

(再掲)4. h4!?まで
\end{center}

チェスは将棋と違って、序盤でルークの先のポーンを伸ばしていくことは非常に珍しいです。それよりもセンターを支配することが重要、とはよく言われることです。

この局面でも、センターの重要なd4, e5マスに効きを増やす4. Nf3は、非常に自然な手です。それに比べると4. h4は不自然な手にも見えます。

それではなぜ、4. h4が指されるのでしょうか?\\

{\Large a. キングサイドにスペースを確保するもっとも直接的な手である}\\

スペースアドバンテージという概念があります。自由に使えるマスの多さ、とも言いかえることができるかと思いますが、自分がピースをその中で自由に動かせる空間が多いというアドバンテージです。

よく言われるのは伸ばしたポーンの内側ですが、それに限らずピースの効きによって相手がピースを置けないマスも自分のスペースと考えることもできます。

少し変化を進めてみましょう。

{\bf 4... h6 5. g4 Bd7 6. h5 e6 7. f4 (Caruana-L'Ami, 2013)}
\begin{center}
\chessboard[setfen=\fenb]

\end{center}

白のキングサイドは、白が好きに駒を配置できます。例えば、Ne2-Ng3-Qf3-Bg2-O-Oとしてf5から強くポーンを押していくこともできるかもしれません。一方黒はどうでしょうか。g8のナイトが動ける先はe7のみ、d7のビショップは今現在c8にしか戻れず、クイーンも相当動きが制限されています。

このように好きに駒を配備できてプランの選択が可能、というのがスペースアドバンテージの利点です。4. h4は次にキングサイドでスペースを確保するという積極的なプランの元にもなっています。\\

{\Large b. キングサイドでタクティカルなチャンスを生める、あるいは黒のポーン形を崩せる}\\

Caro-Kannプレーヤーなら、一度はこのゲームを指したことがあるか、あるいは少なくとも見たことがあると思います。

{\bf 1. e4 c6 2. d4 d5 3. e5 Bf5 4. h4 e6?? 5. g4 Be4 6. f3 Bg6 7. h5}
\begin{center}
\chessboard[setfen=\fenc]

\end{center}

4手目が他の手であれば、4... e6は好手なので、ついうっかりしがちです。

そうでなくとも、f5やg6のビショップをナイトなどによりアタックされることもあります。黒が最もドラスティックに白のキングサイドでの攻勢を防ぐプランは4... h5ですが、今度はh5のポーンが攻撃対象になり、g5のマスは白が好きに使える可能性が増します。

白は黒に「キングサイドを若干弱める」か「自由に白にキングサイドのスペースを取らせてキングサイドでのタクティクスのチャンスを作らせる」か、あるいは「手損する」(f5に出たビショップをd7に引く展開もあります)を選ばせることができます。

このように、4. h4には良い点がいくつかありますが、もちろん悪い点もあります。黒にセンターからのカウンターを許すこと、伸ばしたキングサイドのポーンがターゲットになること、などです。

このように、どちらにも異なった主張があるため、非常にエキサイティングなゲームになります。

4. h4に対する黒の主な受け方は、4...h5、4...h6、4... c5、4... Qb6などがあります。どれも一局ですが、それぞれ全く違った局面になるので、面白いところです。

\subsection{4... h6の変化}
序盤定跡を学ぶときには、相手が最善の手、あるいは最もクリティカルな手を指さなかった場合に自分がどう指すと優勢になるか、を知ることが大事です。メインラインだけを抑えるのは良くないと言われる所以でもあります。

その時に重要になるのが、序盤定跡におけるthematicなプランです。この定跡形ではこの手を狙う、この手を指せば満足、という手を知っていると、「その手を指せるかどうか」という観点で局面を見ることができるため、手の選択にも悩まなくなり、定跡から外れた際の指し方の指針にもなります。

Tal Variation 4. h4の変化で最もクリティカルな手は4... h5と思いますが、それ以外の手に対してはどのように対応していくか、これから見ていきたいと思います。

まずは4... h6です。

{\bf 1. e4 c6 2. d4 d5 3. e5 Bf5 4. h4 h6}
\begin{center}
\chessboard[setfen=\fend]

\end{center}
手の狙いを考える時には、「もし自分が1手パスしたら相手は何を指すか」を考えることが大事です。この局面、もし白が1手パスするならば、黒は5... e6を指すでしょう。

5... e6後は、6. g4とされても形よく6... Bh7と引くことができます。「バッドビショップはポーンチェーンの外側に」と言われますが、まさにそのような形になっています。

とすれば、白は5...e6を許さないような手を指せば、黒のプランを崩すことができます。

それが5. g4!です。

{\bf 5. g4}
\begin{center}
\chessboard[setfen=\fene]

\end{center}
さて、黒はビショップがアタックされている以上、逃げなければなりません。最も自然な手は5... Bh7ですが、成立するでしょうか?

\subsubsection{4... h6 5. g4 Bh7}
{\bf 5. g4 Bh7?! 6. e6!}

この定跡は黒のキングサイドのポーン形を崩すことが一つのテーマになります。そのため、このe6突きは非常に強力です。

{\bf 6... fxe6 7. Bd3 Bxd3 8. Qxd3}

白は、黒の弱いg6, e6マスを攻めたいため、黒の白マスビショップを交換して消してしまいます。

{\bf 8... Qd6 9. f4 Nd7 10. Nf3 O-O-O 11. Ne5 Nxe5 12. fxe5 Qd7 13. h5}

自然に進めるとこのようになります。白のキングサイドのスペースとピースの動かしやすさ、黒のポーン得という構図になりますが、やはり黒のキングサイドが硬直するのが大きく、この局面は白が良いと思います。

これを避けるために、黒は10... e5!としてポーンを返すのが面白いでしょう。しかし、それでもやはりキングサイドのスペースは大きく、白が良いと思います。

\subsubsection{4... h6 5. g4 Bd7}
{\bf 4... h6 5. g4 Bd7}

5... Bd7と、こちらに引くのが良いとされています。これは手損であり、Caro-Kannのテーマである、白マスビショップをポーンチェーンの外に出してからセンターに反撃するというプランとも一貫していないように見えますが、白のキングサイドのポーン突きを緩手にする(攻撃対象をキングサイドから退避させることで、白のPh4-Pg4がキングサイドを弱めただけの手にさせる)という狙いがあります。

さらに、黒はここからe6-c5と指せばフレンチのポーン形になるので、ポーン形は全く問題がありません。

白にはここからいくつか手があるのですが、一つ面白いプランを紹介します。黒が、「e6-c5を指したい」というプランを持っていることに目を付け、このプランを阻止するように指します。

{\bf 6. Nd2!}

Kramnik-Leko(2004)の新手であり、第1回で紹介した、世界選手権最終ラウンドのゲームの手でもあります。

この手には、次に7. Nb3として黒からの...c5を防ぐ狙いがあります。

{\bf 6... c5}

それでも突いてしまいます。他にも6... Qc8などもありますが、白のプランは同じです。最終的にはNb3を狙い、黒の...c5に対して対処します。

{\bf 7. dxc5 e6 8. Nb3}

これで白はポーン得を守れるように見えますが、

{\bf 8... Bxc5! 9. Nxc5 Qa5+ 10. c3 Qxc5}

これでポーンを取り返せます。

{\bf 11. Nf3 Ne7}

11... Qc7もありますが、黒はバッドビショップである白マスビショップを解消することが課題になります。

\begin{center}
\chessboard[setfen=\fenf]

\end{center}
この局面をどう評価するかですが、黒のポーンストラクチャーはコンパクトで好形ですがバッドビショップを持っています。一方白もダブルビショップを持っていますが、キングサイドのポーン形が崩れています。アグレッシブなプレーを好むプレーヤーは白を、ソリッドなプレーヤーは黒を持ちたいと考える局面と思います。

この後ですが、12. Nd4が強いように感じます。黒はバッドビショップを解消するために...Bb5からのビショップ交換が一つの狙いになるため、b5マスを抑えてしまう狙いの手です。実戦例は12. Bd3か12. h5ですが、例えば12. Nd4 Nbc6 13. Nb3 Qb6 14. Be3 Qc7 15. f4と進めて、センターの黒マスを支配すれば白は指しやすいように思います。

\subsection{4... Qb6の変化}
4... Qb6は一種の手待ちであり、キングサイドを4... h5や4... h6で弱めずに、白のg4突きに対してBd7に引くため、eポーンもhポーンも動かさない、という手です。

加えて、白のd4ポーンに若干の圧力をかけた上で、...c5を準備しています。

{\bf 1. e4 c6 2. d4 d5 3. e5 Bf5 4. h4 Qb6}

\def\feng{rn2kbnr/pp2pppp/1qp5/3pPb2/3P3P/8/PPP2PP1/RNBQKBNR w KQkq - 1 5}
\begin{center}
\chessboard[setfen=\feng]

\end{center}

白にはいくつかの手がありますが、もっとも直接的な5. g4には5... Bd7と引いておいて、黒はフレンチ風に戦えます。ここでdポーンが当たっているためにKramnikの6. Nd2が指せないのがポイント(6. Nd2?? Qxd4 -+)です。

また、5. a4も面白い手ですが、そのような手があることの紹介にとどめ、深くは追いません。

{\bf 5. Nc3}

おそらく4... Qb6に対して最も効果的なのはこの手です。直接的には5... c5を防ぐ意味があります(5... c5?? 6. Nxd5)が、黒は白マスビショップ問題を何とかしないとeポーンを突けないため、c5を突くことができなくなります。

{\bf 5... h5}

結局黒は、hポーンを突くことになりました。それでは、4... h5と同じ変化になるのではないかと思う向きもあるとは思いますが、一つ大きな違いがあります。

次回紹介しますが、4... h5に対しては5. c4から6. Nc3と、「Pc4-Nc3型」を作るのが白としては効果的です。しかし、この手順では5. Nc3と先に跳ねているため、「Pc2-Nc3型」を白は強いられています。

このことにより、黒はややc5を突きやすくなっていると言えるでしょう。

{\bf 6. Nge2}

次回詳しく紹介しますが、この手はNg3からBe2として、h5をターゲットにしていく狙いがあります。黒の4... Qb6のおかげで、白はh4をターゲットにされにくい形になっています。

{\bf 6... e6 7. Ng3 Bg6 8. Be2 c5}

お互いに、白はh5をターゲットにする、黒は...c5からセンターをブレイクするという、当初の目的を達成したことになります。

{\bf 9. dxc5}

f8のビショップが動いていないときのこのような手はフレンチやカロカンではあまり好ましくはないですが、仕方がありません。

{\bf 9... Bxc5 10. O-O}

\def\fenh{rn2k1nr/pp3pp1/1q2p1b1/2bpP2p/7P/2N3N1/PPP1BPP1/R1BQ1RK1 b kq - 1 10}
\begin{center}
\chessboard[setfen=\fenh]

\end{center}
ひと段落しました。黒はh5のポーンを除いては大きな弱点がなく、ビショップも活動的です。白はe5, b2, h4のポーンがターゲットになりやすい陣形ですが、ナイトがセンターに効いており次にNa4の狙いもあります。ダイナミックなチャンスは白にありますが、ポジションとしては黒十分でしょう。ここから、白がどうやって手を作っていくか、白が考える必要があります。

{\bf 10... Be7}

Na4の狙いを受けつつ、h4のポーンに狙いをつける自然な手です。

{\bf 11. Nb5!}

c7とd6を睨むことで、黒に11... Bxh4と指しづらくする手です。

{\bf 11... a6 12. Be3 Qd8 13. Nd4}

やはりd4が好位置です。何かの拍子にe6にサクリファイスすることも視野に入れられます。

この後は、Tindall - Smith (2002, Oceania Zonal Tournament)の進行をなぞります。

{\bf 13... Bxh4 14. Nxh5 Bxh5 15. Bxh5 Qd7}

白にNxe6の狙いがありました。

{\bf 16. g3 Bd8 17. f4 g6 18. Bf3 =}

この局面はもろもろの要素を考えて、白がやや良し(序盤のアドバンテージを失っていないくらい)と考えられます。

4... Qb6は、いったん別の手を白に指させたのちメインラインに近い形に戻すことで、変化を限定しているという意味で面白い手であると思います。

\subsection{4... h5の変化}
{\bf 1. e4 c6 2. d4 d5 3. e5 Bf5 4. h4 h5}
\def\feni{rn1qkbnr/pp2ppp1/2p5/3pPb1p/3P3P/8/PPP2PP1/RNBQKBNR w KQkq h6 0 5}
\begin{center}
\chessboard[setfen=\feni]

\end{center}
黒は白のキングサイドの拡張を止めるために、若干キングサイドを弱めます。今まで他の手を見てきたときに、白の狙いの一つはg4突きであるということを強調してきましたが、この手はもっとも直接的に5. g4を防いでいます。

白は、黒のh5のポーンをターゲットにして指していきたいところです。そのためにナイトの動きとして、Ne2-Ng3を考えます。このナイトの動きはTal Variationの白番に特有の動きであり、この形からh5を取ることを狙いにして指していくことになります。

メインラインは5. c4ですが、その前に直接白がh5のポーンを取りに行くとどうなるかを見てみます。

\subsubsection{4... h5 5. Ne2}
{\bf 1. e4 c6 2. d4 d5 3. e5 Bf5 4. h4 h5 5. Ne2!?}

狙いはシンプルに、Ng3-Be2としてh5のポーンを取りに行くことです。黒はNf6とできないため、h5の数がどうやっても足りず、白がポーン得するか、Bh7-Pg6型を強要できるように見えますが……

{\bf 5... e6 6. Ng3 Bg6 7. Be2}
\def\fenj{rn1qkbnr/pp3pp1/2p1p1b1/3pP2p/3P3P/6N1/PPP1BPP1/RNBQK2R b KQkq - 0 7}
\begin{center}
\chessboard[setfen=\fenj]

\end{center}
黒の反撃は?

{\bf 7... c5!}

いかにもCaro-Kannらしい反撃で、サイドからの攻撃に対してはセンターで反撃すべし、という原則にも従っています。手を進めてみます。

{\bf 8. c3 Nc6 9. Be3 Qb6 10. Qb3 c4 11. Qxb6 axb6}

対ロンドンの黒番定跡や、フレンチの黒番定跡になじみ深いプランで、黒が有利です。この後はb5-b4を狙っていきます。

このような反撃があるため、白の狙いとして、キングサイドを狙っていく前にセンターを固定することが有効です。

\subsubsection{4... h5 5. c4}
5. Ne2からh5のポーンをいきなりアタックすると失敗するので、白はまずはセンターを固定化する必要があります。こう見ていくと、メインラインの5. c4の一つの狙いが見えてきます。

{\bf 1. e4 c6 2. d4 d5 3. e5 Bf5 4. h4 h5 5. c4}
\def\fenk{rn1qkbnr/pp2ppp1/2p5/3pPb1p/2PP3P/8/PP3PP1/RNBQKBNR w KQkq - 0 1}
\begin{center}
\chessboard[setfen=\fenk]

\end{center}
この手は6. cxd5 cxd5として黒からの...c5反撃を防ぎつつ、キングサイドで圧力をかけていく狙いがあります。

{\bf 5... e6}

最も自然な手です。

{\bf 6. Nc3}

黒のd5ポーンに圧力をかけながら...Bxb1としてバッドビショップを解消される手を事前に受けます。

ここで黒は手が広く、6... Ne7、6... Nd7、6... Be7、6...dxc4などが指されています。
「白はcxd5に対して黒にcxd5と取り返させるのが狙い」ということを抑えていると、このあたりの黒の指し手の指針となるでしょう。

\paragraph{6... Nd7}
\mbox{}\newline
6... Nd7は非常にCaro-Kannらしく自然な手ですが、この場合に限っては白に好手段があります。とはいえその手はすでに予告していますが。

{\bf 7. cxd5!}

センターの緊張を解消する手ですが、白はf5のビショップやh5のポーンをターゲットにしてキングサイドで手を作れるため、手に困ることがありません。

{\bf 7... cxd5}

...c5の狙いを残す7... exd5は、8. Bd3 Bxd3 9. Qxd3からNf3-Bg5-Pe6などの組み合わせで、黒のキングサイドが修復できないほどダメージを受けます。

{\bf 8. Bg5}

8. Nge2からNg3-Be2を狙う手はまだ成立しません。詳細は省きますが、黒がNe7-Nc6-Ndxe5とする反撃があり、黒がセンターを逆に支配します。

{\bf 8... Be7 9. Qd2 a6}

こうやってe7をビショップで埋めさせた後に、

{\bf 10. Nge2!}

ようやく当初のプランを実行します。

{\bf 10... Rc8 11. Ng3 Bg6 12. Be2 Bxg5 13. hxg5}

\def\fenl{2rqk1nr/1p1n1pp1/p3p1b1/3pP1Pp/3P4/2N3N1/PP1QBPP1/R3K2R b KQk - 0 13}
\begin{center}
\chessboard[setfen=\fenl]

\end{center}

この局面は、キングサイドの圧力の強さと、ビショップの働きの差で先手が指しやすいでしょう。

\paragraph{6... dxc4}
\mbox{}\newline
黒からセンターの緊張を解消する手で、白のd4ポーンをバックワードポーンにしてターゲットにするという意味もあります。その代わり、白は相手の手に乗って展開ができます。

{\bf 6... cxd4 7. Bxc4 Nd7}
\def\fenm{r2qkbnr/pp1n1pp1/2p1p3/4Pb1p/2BP3P/2N5/PP3PP1/R1BQK1NR w KQkq - 0 8}
\begin{center}
\chessboard[setfen=\fenm]

\end{center}

前回の6... Nd7と、今回の7... Nd7の違いは明白です。前回は白からcxd5と取られると、黒からの...c5が不可能になりました。今回は、先にセンターを解消しているため、黒からの...c5がオプションとして残ります。

{\bf 8. Nge2 Nb6 9. Bb3 Be7 10.Ng3 Bg6}

4. h4のもう一つのポイントは、g5マスを白が使いやすくなることです。9... Be7は、g5マスを抑えつつ、h4のポーンにも狙いを付けています。

{\bf 11. Nge4}

黒がh4のポーンを取ると、Nd6+が非常に厳しいです。4. h4の形のもう一つのポイントで、Ng3-Ne4-Nd6というルートを見せることで黒の駒組みを制限します。

{\bf 11... Nh6 12. Bxh6 Rxh6 13. Qd2}

この局面は、黒も十分やれるという評価をされているようです。

代えて10. g3!が白としては別プランです。

\paragraph{6... Be7}
\mbox{}\newline
この手は白のh4にプレッシャーをかけつつg5を抑える狙いです。

{\bf 6... Be7 7. cxd5 cxd5}

白はこれで満足なように見えますが、h4がアタックされているのでうまくNge2-Ng3ができません。

{\bf 8. Bd3 Bxd3 9. Qxd3 Nc6 10. Nf3 Rc8 11. g3}
\def\feno{2rqk1nr/pp2bpp1/2n1p3/3pP2p/3P3P/2NQ1NP1/PP3P2/R1B1K2R b KQk - 0 11}
\begin{center}
\chessboard[setfen=\feno]

\end{center}

お互いに相手の狙いを受けることで、より穏やかな局面になります。チャンスは互角と思います。

\paragraph{6... Ne7}
\mbox{}\newline
{\bf 1. e4 c6 2. d4 d5 3. e5 Bf5 4. h4 h5 5. c4 e6 6. Nc3 Ne7}

\def\fenp{rn1qkb1r/pp2npp1/2p1p3/3pPb1p/2PP3P/2N5/PP3PP1/R1BQKBNR w KQkq - 2 7}
\begin{center}
\chessboard[setfen=\fenp]

\end{center}

この手は非常に強力です。白から7. cxd5としてセンターを固めてしまう手を防ぐとともに、f5に飛ぶ手を見せます。

ここから白はいろいろなプランがありますが、7. cxd5は7... Nxd5!で互角になります。また、7. Bg5もあります。

最近流行りなのは7. Nge2なので、この手を見ていきましょう。

\subparagraph{7. Nge2 dxc4}
\mbox{}\newline
{\bf 7. Nge2 dxc4}

一旦cポーンが浮くので、この手もあります。

{\bf 8. Ng3 b5}

7... dxc4と指したからには当然ポーンを守りたいところです。この時9. Nxf5?に対して9... Nxf5!と形よく取れるのも、6... Ne7の効用です。

{\bf 9. Bg5 Qa5}

4... h5の一つの欠点としては、g5のコントロールが弱くなることです。そのため、9. Bg5としてg5マスを使いにいくことが有効になります。9... Qa5はビショップのピンを外しつつc3のナイトの動きに制限をかける手ですが、

{\bf 10. a4! b4 11. Nce4 Bxe4 12. Nxe4 Nf5!}

何回かテーマになっていた、d6へのナイトの飛び込みも、これで防げます。

{\bf 13. Bxc4 +=}

ポーンを取り返し、やや白が良いでしょう。黒は9... Qa5に代えて、9... Qb6や9... Qd7などを模索する必要があると思います。

\subparagraph{7. Nge2 Nd7}
\mbox{}\newline
白のcxd5が効果的ではないようにしてから、...Nd7から...c5を決行する狙いです。

{\bf 7. Nge2 Nd7 8. Ng3 Bg6 9. Bg5}

やはり、このピンは強力です。

{\bf 9... Qb6 10. Rc1!?}

\def\fenq{r3kb1r/pp1nnpp1/1qp1p1b1/3pP1Bp/2PP3P/2N3N1/PP3PP1/2RQKB1R b Kkq - 0 10}
\begin{center}
\chessboard[setfen=\fenq]

\end{center}

b2を狙う黒の9手目に対し、d2に上がって受ける10. Qd2もありましたが、その後の展開は黒良しとされています。代えて、Sutovskyによる10. Rc1!?が調べられています。

ポイントは、いったんc3のナイトを守ることで、10... Qxb2に対しては11. Bd3とするテンポを稼ぎ、Rb1を狙いに指していくことです。

一例として10... Qxb2 11. Bd3 dxc4 12. Bxg6! Nxg6 13. O-O Qa3 14. Nxh5のように進み、白はギャンビットしたポーン分の代償がある局面でしょう。この局面はまだそこまで研究が進んでおらず、b2のポーンを取れるのか、取れないとしたら白が良いのか、は難しい局面と思います。