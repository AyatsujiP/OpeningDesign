\section*{はじめに}
\addcontentsline{toc}{section}{はじめに}

本書は、著者のブログに連載していた、チェスの序盤定跡に関する記事に大幅に加筆修正を行ったものです。

日本でチェスを指す際に常に問題になっていた(そして、現在でも決して解決されているとはいいがたい)ことは、日本語による情報の少なさでした。近年でこそ、FischerのMy 60 memorable gamesの和訳など、中級者、上級者向けの本が公刊され始めていますが、それまではチェス書籍といえば入門書ばかり、といった状況にありました。このような、「入門書を読み終えたプレーヤーが日本でチェスを勉強するには英語が必須」といわれる状況は、チェスの普及にとって、好ましい状況とはいいがたいと考えています。
私も一介のチェスプレーヤーとして、日本のチェス人口がもっと増えてほしい、と夙に願っています。そのためにできることとして、私はブログにおいて大会のレポートや序盤の紹介を行ってきました。日本でチェスを続けていきたい、もっと勉強したいと思う方のために、少しでも情報発信を行いたいと思い、ブログで情報発信を続けてきました。

ここに、ブログで書いてきた記事のうち、序盤定跡に関する記事を切り出し、加筆修正してpdfの形でまとめて読めるように編集しなおします。ブログの形式であると複数の記事に分かれることによる読みにくさもあったため、読みやすさ・見やすさを優先して編集を行います。

チェスの序盤、そして序盤研究は非常に面白い分野です。コンピュータ、そしてチェスソフトが発展した現代にあっても、その面白さは減っていないと思っています。序盤のわずかな形の違いが、中盤でのプランの違い、そして終盤での勝敗の違いにつながることも決して少なくはないでしょう。そのようなわずかな形の違いに気づくこと、その違いが何を意味するのかを考えることは、極めて論理的な作業であるとともに創造的な作業でもあります。

序盤研究はコンピュータ、およびデータベースを使うことが一般的になっています。しかし、序盤を研究することは、決して手順を暗記することではありません。なぜコンピュータがその手順を最善としているのか、なぜスーパーグランドマスターがその手を指すのか、といった意味を知る必要があります。そしてその意味は、定跡ごとの狙いと、手順による形の違いを知り、そこからのお互いが可能なプランの違いを考えることにより、局面が教えてくれるものです。

本書では、1. e4に対する黒の代表的な応答(1... c6, 1... e6, 1... c5, 1...e5, その他)を取り上げます。本書は、分岐する序盤全てを取り扱うわけではなく、いわゆる「オープニングツリー」を作ることが本書の目的ではありません。そうではなく、序盤を研究する中で何をポイントにして研究するか、を説明する例として、いくつかの定跡を取り上げていると考えてほしいと思います。序盤定跡のレパートリーを見直す際の考え方の一助となれば幸いです。

本書は、一通り駒の動かし方や簡単なタクティクスについてすでに学び、対人戦においても何度か勝つことができるようになったレベルのプレーヤーから、FIDEレーティング1700前後のプレーヤーまでを想定読者としています。もし、本書の中でわからない単語があった場合には、以下のページが参考になるでしょう。\\
チェス用語小辞典(英和)\\
http://hnishy.la.coocan.jp/chessterms.htm
